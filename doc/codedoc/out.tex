\documentclass[a4paper,11pt]{report}
\usepackage{anysize}
\usepackage{graphicx}
\usepackage{natbib}
\usepackage{amsmath, amsthm, amssymb}
\usepackage{setspace}
\usepackage{units}
\usepackage[small,bf]{caption}
\usepackage{fancyhdr}
\usepackage{color}
\usepackage{hyperref}

\onehalfspacing

\hypersetup{
    bookmarks=true,         % show bookmarks bar?
    unicode=false,          % non-Latin characters in Acrobat�s bookmarks
    pdfborder={0 0 0 0}
    pdftoolbar=true,        % show Acrobat�s toolbar?
    pdfmenubar=true,        % show Acrobat�s menu?
    pdffitwindow=false,     % window fit to page when opened
    pdfstartview={FitH},    % fits the width of the page to the window
    pdftitle={My title},    % title
    pdfauthor={Author},     % author
    pdfsubject={Subject},   % subject of the document
    pdfcreator={Creator},   % creator of the document
    pdfproducer={Producer}, % producer of the document
    pdfkeywords={keywords}, % list of keywords
    pdfnewwindow=true,      % links in new window
    colorlinks=true,       % false: boxed links; true: colored links
    linkcolor=blue,          % color of internal links
    citecolor=green,        % color of links to bibliography
    filecolor=magenta,      % color of file links
    urlcolor=blue           % color of external links
}
\pagestyle{fancy} \fancyhead{} \fancyfoot{} \fancyhead[LO] {\leftmark} \fancyhead[RO]{\thepage}


\begin{document}


\tableofcontents
\chapter*{Classes}
\addcontentsline{toc}{chapter}{Classes}
\section*{SDIAutoParallel \label{SDIAutoParallel__define_lab}}
\addcontentsline{toc}{section}{SDIAutoParallel}
No Doc\newline \newline
Inherits from: \textbf{XDIBASE} \newline
Class Data: 
\begin{table}[!h]
\begin{tiny}\vspace{-.1cm}\begin{center}
\begin{tabular}{rl|rl|rl}
\hline
(\verb"long") & id& (\verb"string") & status& (\verb"float") & wavelength \\
(\verb"double") & start\_time& (\verb"float") & param& (\verb"int") & step \\
(\verb"int") & nominal& (\verb"int") & leg1& (\verb"int") & leg2 \\
(\verb"int") & leg3& (\verb"int") & curr\_leg& (\verb"int") & param\_pos \\
(\verb"ptr") & ref\_image& (\verb"int") & get\_ref\_flag& (\verb"string") & obj\_num \\
(\verb"structure") & geometry& (\verb"int") & need\_frame& (\verb"int") & need\_timer \\
(\verb"int") & auto& (\verb"structure") & palette& (\verb"obj") & manager \\
\hline
\end{tabular}\end{center}\end{tiny}\end{table}\vspace{-.5cm} \\
\textbf{Defined in file:} \newline
\small{C:/cal/Operations/SDI\_Instruments/common/idl/core/sdiautoparallel\_\_define.pro}
\begin{center}\rule{1\textwidth}{.02cm}\end{center}
\subsection*{METHODS:}
\subsubsection*{(function) INIT \label{SDIAutoParallel::init_lab}}
\textbf{Method Documentation:} \\
No Doc\newline
\textbf{Arguments:}
\begin{description}
\vspace{-.15cm}
\item[] \hspace{.5cm} \verb"data=data": No Doc
\vspace{-.15cm}
\item[] \hspace{.5cm} \verb"restore_struc=restore_struc": No Doc
\end{description}
Example Call:
\begin{align*}
result = \mathbf{SDIAutoParallel}-\hspace{-.15cm}>\mathbf{init}(&data=data,\\ \ &restore_struc=restore_struc)
\end{align*}
\begin{center}\rule{.85\textwidth}{.01cm}\end{center}
\subsubsection*{(pro) CLEANUP \label{SDIAutoParallel::cleanup_lab}}
\textbf{Method Documentation:} \\
No Doc\newline
\textbf{Arguments:}
\begin{description}
\vspace{-.15cm}
\item[] \hspace{.5cm} \verb"log": No Doc
\end{description}
Example Call:
\begin{align*}
\mathbf{SDIAutoParallel}-\hspace{-.15cm}>\mathbf{cleanup}, \ &log
\end{align*}
\begin{center}\rule{.85\textwidth}{.01cm}\end{center}
\subsubsection*{(pro) FRAME\_EVENT \label{SDIAutoParallel::frame_event_lab}}
\textbf{Method Documentation:} \\
No Doc\newline
\textbf{Arguments:}
\begin{description}
\vspace{-.15cm}
\item[] \hspace{.5cm} \verb"image": No Doc
\vspace{-.15cm}
\item[] \hspace{.5cm} \verb"channel": No Doc
\end{description}
Example Call:
\begin{align*}
\mathbf{SDIAutoParallel}-\hspace{-.15cm}>\mathbf{frame\_event}, \ &image,\\ \ &channel
\end{align*}
\begin{center}\rule{.85\textwidth}{.01cm}\end{center}
\subsubsection*{(function) GET\_SETTINGS \label{SDIAutoParallel::get_settings_lab}}
\textbf{Method Documentation:} \\
No Doc\newline
Takes no arguments \\
Example Call:
\begin{align*}
result = \mathbf{SDIAutoParallel}-\hspace{-.15cm}>\mathbf{get\_settings}(&)
\end{align*}
\begin{center}\rule{.85\textwidth}{.01cm}\end{center}
\subsubsection*{(pro) START\_PARALLEL \label{SDIAutoParallel::start_parallel_lab}}
\textbf{Method Documentation:} \\
No Doc\newline
\textbf{Arguments:}
\begin{description}
\vspace{-.15cm}
\item[] \hspace{.5cm} \verb"event": No Doc
\end{description}
Example Call:
\begin{align*}
\mathbf{SDIAutoParallel}-\hspace{-.15cm}>\mathbf{start\_parallel}, \ &event
\end{align*}
\begin{center}\rule{.85\textwidth}{.01cm}\end{center}
\subsubsection*{(pro) STOP\_PARALLEL \label{SDIAutoParallel::stop_parallel_lab}}
\textbf{Method Documentation:} \\
No Doc\newline
\textbf{Arguments:}
\begin{description}
\vspace{-.15cm}
\item[] \hspace{.5cm} \verb"event": No Doc
\end{description}
Example Call:
\begin{align*}
\mathbf{SDIAutoParallel}-\hspace{-.15cm}>\mathbf{stop\_parallel}, \ &event
\end{align*}
\begin{center}\rule{.85\textwidth}{.01cm}\end{center}
\newpage
\section*{SDIEtalonScanner \label{SDIEtalonScanner__define_lab}}
\addcontentsline{toc}{section}{SDIEtalonScanner}
The EtalonScanner plugin lets you continuously scan the etalon over one order of interference at a given wavelength, and optionally pause during a scan.\newline \newline
Inherits from: \textbf{XDIBASE} \newline
Class Data: 
\begin{table}[!h]
\begin{tiny}\vspace{-.1cm}\begin{center}
\begin{tabular}{rl|rl|rl}
\hline
(\verb"long") & id& (\verb"string") & status& (\verb"float") & wavelength \\
(\verb"double") & start\_time& (\verb"int") & nchann& (\verb"string") & obj\_num \\
(\verb"structure") & geometry& (\verb"int") & need\_frame& (\verb"int") & need\_timer \\
(\verb"int") & auto& (\verb"structure") & palette& (\verb"obj") & manager \\
\hline
\end{tabular}\end{center}\end{tiny}\end{table}\vspace{-.5cm} \\
\textbf{Defined in file:} \newline
\small{C:/cal/Operations/SDI\_Instruments/common/idl/core/sdietalonscanner\_\_define.pro}
\begin{center}\rule{1\textwidth}{.02cm}\end{center}
\subsection*{METHODS:}
\subsubsection*{(function) INIT \label{SDIEtalonScanner::init_lab}}
\textbf{Method Documentation:} \\
Initialize the EtalonScanner.\newline
\textbf{Arguments:}
\begin{description}
\vspace{-.15cm}
\item[] \hspace{.5cm} \verb"data=data": Misc data
\vspace{-.15cm}
\item[] \hspace{.5cm} \verb"restore_struc=restore_struc": Restored settings
\end{description}
Example Call:
\begin{align*}
result = \mathbf{SDIEtalonScanner}-\hspace{-.15cm}>\mathbf{init}(&data=data,\\ \ &restore_struc=restore_struc)
\end{align*}
\begin{center}\rule{.85\textwidth}{.01cm}\end{center}
\subsubsection*{(pro) CLEANUP \label{SDIEtalonScanner::cleanup_lab}}
\textbf{Method Documentation:} \\
Cleanup, stop any current scans.\newline
\textbf{Arguments:}
\begin{description}
\vspace{-.15cm}
\item[] \hspace{.5cm} \verb"log": No Doc
\end{description}
Example Call:
\begin{align*}
\mathbf{SDIEtalonScanner}-\hspace{-.15cm}>\mathbf{cleanup}, \ &log
\end{align*}
\begin{center}\rule{.85\textwidth}{.01cm}\end{center}
\subsubsection*{(pro) FRAME\_EVENT \label{SDIEtalonScanner::frame_event_lab}}
\textbf{Method Documentation:} \\
A new frame has been recieved. Update leg diagrams, decide if we need to start a new scan.\newline
\textbf{Arguments:}
\begin{description}
\vspace{-.15cm}
\item[] \hspace{.5cm} \verb"image": The new camera frame
\vspace{-.15cm}
\item[] \hspace{.5cm} \verb"channel": The current scan channel
\end{description}
Example Call:
\begin{align*}
\mathbf{SDIEtalonScanner}-\hspace{-.15cm}>\mathbf{frame\_event}, \ &image,\\ \ &channel
\end{align*}
\begin{center}\rule{.85\textwidth}{.01cm}\end{center}
\subsubsection*{(function) GET\_SETTINGS \label{SDIEtalonScanner::get_settings_lab}}
\textbf{Method Documentation:} \\
Select settings to save.\newline
Takes no arguments \\
Example Call:
\begin{align*}
result = \mathbf{SDIEtalonScanner}-\hspace{-.15cm}>\mathbf{get\_settings}(&)
\end{align*}
\begin{center}\rule{.85\textwidth}{.01cm}\end{center}
\subsubsection*{(pro) PAUSE\_SCAN \label{SDIEtalonScanner::pause_scan_lab}}
\textbf{Method Documentation:} \\
Pause the current scan.\newline
\textbf{Arguments:}
\begin{description}
\vspace{-.15cm}
\item[] \hspace{.5cm} \verb"event": Widget event
\end{description}
Example Call:
\begin{align*}
\mathbf{SDIEtalonScanner}-\hspace{-.15cm}>\mathbf{pause\_scan}, \ &event
\end{align*}
\begin{center}\rule{.85\textwidth}{.01cm}\end{center}
\subsubsection*{(pro) SET\_WAVELENGTH \label{SDIEtalonScanner::set_wavelength_lab}}
\textbf{Method Documentation:} \\
Set the wavelength for scanning.\newline
\textbf{Arguments:}
\begin{description}
\vspace{-.15cm}
\item[] \hspace{.5cm} \verb"event": Widget event
\end{description}
Example Call:
\begin{align*}
\mathbf{SDIEtalonScanner}-\hspace{-.15cm}>\mathbf{set\_wavelength}, \ &event
\end{align*}
\begin{center}\rule{.85\textwidth}{.01cm}\end{center}
\subsubsection*{(pro) START\_SCAN \label{SDIEtalonScanner::start_scan_lab}}
\textbf{Method Documentation:} \\
Start a scan.\newline
\textbf{Arguments:}
\begin{description}
\vspace{-.15cm}
\item[] \hspace{.5cm} \verb"event": Widget event
\end{description}
Example Call:
\begin{align*}
\mathbf{SDIEtalonScanner}-\hspace{-.15cm}>\mathbf{start\_scan}, \ &event
\end{align*}
\begin{center}\rule{.85\textwidth}{.01cm}\end{center}
\subsubsection*{(pro) STOP\_SCAN \label{SDIEtalonScanner::stop_scan_lab}}
\textbf{Method Documentation:} \\
Stop the current scan (will restart from beginning on next `start')\newline
\textbf{Arguments:}
\begin{description}
\vspace{-.15cm}
\item[] \hspace{.5cm} \verb"event": Widget event
\end{description}
Example Call:
\begin{align*}
\mathbf{SDIEtalonScanner}-\hspace{-.15cm}>\mathbf{stop\_scan}, \ &event
\end{align*}
\begin{center}\rule{.85\textwidth}{.01cm}\end{center}
\newpage
\section*{SDIEtalonSpacer \label{SDIEtalonSpacer__define_lab}}
\addcontentsline{toc}{section}{SDIEtalonSpacer}
The EtalonSpacer plugin allows you to adjust the etalon plate separation at each leg. You can control each leg individually, or adjust paralellism along two orthogonal axes.\newline \newline
Inherits from: \textbf{XDIBASE} \newline
Class Data: 
\begin{table}[!h]
\begin{tiny}\vspace{-.1cm}\begin{center}
\begin{tabular}{rl|rl|rl}
\hline
(\verb"long") & id& (\verb"string") & status& (\verb"int") & step \\
(\verb"int") & leg1& (\verb"int") & leg2& (\verb"int") & leg3 \\
(\verb"string") & obj\_num& (\verb"structure") & geometry& (\verb"int") & need\_frame \\
(\verb"int") & need\_timer& (\verb"int") & auto& (\verb"structure") & palette \\
\hline
\end{tabular}\end{center}\end{tiny}\end{table}\vspace{-.5cm} \\
\textbf{Defined in file:} \newline
\small{C:/cal/Operations/SDI\_Instruments/common/idl/core/sdietalonspacer\_\_define.pro}
\begin{center}\rule{1\textwidth}{.02cm}\end{center}
\subsection*{METHODS:}
\subsubsection*{(function) INIT \label{SDIEtalonSpacer::init_lab}}
\textbf{Method Documentation:} \\
EtalonSpacer initialization.\newline
\textbf{Arguments:}
\begin{description}
\vspace{-.15cm}
\item[] \hspace{.5cm} \verb"data=data": Misc data
\vspace{-.15cm}
\item[] \hspace{.5cm} \verb"restore_struc=restore_struc": Restored settings
\end{description}
Example Call:
\begin{align*}
result = \mathbf{SDIEtalonSpacer}-\hspace{-.15cm}>\mathbf{init}(&data=data,\\ \ &restore_struc=restore_struc)
\end{align*}
\begin{center}\rule{.85\textwidth}{.01cm}\end{center}
\subsubsection*{(pro) ADJUST\_LEGS\_EVENT \label{SDIEtalonSpacer::adjust_legs_event_lab}}
\textbf{Method Documentation:} \\
An event from the widget sloders representing leg voltages.\newline
\textbf{Arguments:}
\begin{description}
\vspace{-.15cm}
\item[] \hspace{.5cm} \verb"event": Widget event
\end{description}
Example Call:
\begin{align*}
\mathbf{SDIEtalonSpacer}-\hspace{-.15cm}>\mathbf{adjust\_legs\_event}, \ &event
\end{align*}
\begin{center}\rule{.85\textwidth}{.01cm}\end{center}
\subsubsection*{(pro) CLEANUP \label{SDIEtalonSpacer::cleanup_lab}}
\textbf{Method Documentation:} \\
Cleanup - nothing to do\newline
\textbf{Arguments:}
\begin{description}
\vspace{-.15cm}
\item[] \hspace{.5cm} \verb"log": No Doc
\end{description}
Example Call:
\begin{align*}
\mathbf{SDIEtalonSpacer}-\hspace{-.15cm}>\mathbf{cleanup}, \ &log
\end{align*}
\begin{center}\rule{.85\textwidth}{.01cm}\end{center}
\subsubsection*{(function) GET\_SETTINGS \label{SDIEtalonSpacer::get_settings_lab}}
\textbf{Method Documentation:} \\
Get settings for saving.\newline
Takes no arguments \\
Example Call:
\begin{align*}
result = \mathbf{SDIEtalonSpacer}-\hspace{-.15cm}>\mathbf{get\_settings}(&)
\end{align*}
\begin{center}\rule{.85\textwidth}{.01cm}\end{center}
\subsubsection*{(pro) STEP\_CHANGE \label{SDIEtalonSpacer::step_change_lab}}
\textbf{Method Documentation:} \\
Change the size of the tilt adjustment.\newline
\textbf{Arguments:}
\begin{description}
\vspace{-.15cm}
\item[] \hspace{.5cm} \verb"event": Widget event
\end{description}
Example Call:
\begin{align*}
\mathbf{SDIEtalonSpacer}-\hspace{-.15cm}>\mathbf{step\_change}, \ &event
\end{align*}
\begin{center}\rule{.85\textwidth}{.01cm}\end{center}
\subsubsection*{(pro) TILT \label{SDIEtalonSpacer::tilt_lab}}
\textbf{Method Documentation:} \\
A tilt event, for adjusting along the two orthogonal axes.\newline
\textbf{Arguments:}
\begin{description}
\vspace{-.15cm}
\item[] \hspace{.5cm} \verb"event": Widget event
\end{description}
Example Call:
\begin{align*}
\mathbf{SDIEtalonSpacer}-\hspace{-.15cm}>\mathbf{tilt}, \ &event
\end{align*}
\begin{center}\rule{.85\textwidth}{.01cm}\end{center}
\newpage
\section*{SDIPhaseMapper \label{SDIPhaseMapper__define_lab}}
\addcontentsline{toc}{section}{SDIPhaseMapper}
The Phasemapper plugin records `phase maps' which encode the scan channel at which a spectrum recorded at the phasemap wavelength peaks for every pixel in the camera frame.\newline \newline
Inherits from: \textbf{XDIBASE} \newline
Class Data: 
\begin{table}[!h]
\begin{tiny}\vspace{-.1cm}\begin{center}
\begin{tabular}{rl|rl|rl}
\hline
(\verb"long") & id& (\verb"int") & nscans& (\verb"int") & current\_scan \\
(\verb"int") & scanning& (\verb"int") & nchann& (\verb"float") & wavelength \\
(\verb"int") & channel& (\verb"ptr") & image& (\verb"ptr") & phasemap \\
(\verb"int") & xdim& (\verb"int") & ydim& (\verb"ptr") & p \\
(\verb"ptr") & q& (\verb"ptr") & px& (\verb"ptr") & qx \\
(\verb"int") & source\_order& (\verb"float") & source\_lambda& (\verb"ptr") & source\_pmap \\
(\verb"int") & current\_source& (\verb"float") & gain& (\verb"float") & exptime \\
(\verb"float") & smooth\_window& (\verb"string") & obj\_num& (\verb"structure") & geometry \\
(\verb"int") & need\_frame& (\verb"int") & need\_timer& (\verb"int") & auto \\
(\verb"structure") & palette& (\verb"obj") & manager& (\verb"obj") & console \\
\hline
\end{tabular}\end{center}\end{tiny}\end{table}\vspace{-.5cm} \\
\textbf{Defined in file:} \newline
\small{C:/cal/Operations/SDI\_Instruments/common/idl/core/sdiphasemapper\_\_define.pro}
\begin{center}\rule{1\textwidth}{.02cm}\end{center}
\subsection*{METHODS:}
\subsubsection*{(function) INIT \label{SDIPhaseMapper::init_lab}}
\textbf{Method Documentation:} \\
Phasemapper initialization.\newline
\textbf{Arguments:}
\begin{description}
\vspace{-.15cm}
\item[] \hspace{.5cm} \verb"restore_struc=restore_struc": Misc data
\vspace{-.15cm}
\item[] \hspace{.5cm} \verb"data=data": Restored settings
\end{description}
Example Call:
\begin{align*}
result = \mathbf{SDIPhaseMapper}-\hspace{-.15cm}>\mathbf{init}(&restore_struc=restore_struc,\\ \ &data=data)
\end{align*}
\begin{center}\rule{.85\textwidth}{.01cm}\end{center}
\subsubsection*{(function) AUTO\_START \label{SDIPhaseMapper::auto_start_lab}}
\textbf{Method Documentation:} \\
Auto start the Phasemapper - called whn running in auto mode, and plugin is started from a scheduled command.\newline
\textbf{Arguments:}
\begin{description}
\vspace{-.15cm}
\item[] \hspace{.5cm} \verb"args": String of arguments passed from the schedule file
\end{description}
Example Call:
\begin{align*}
result = \mathbf{SDIPhaseMapper}-\hspace{-.15cm}>\mathbf{auto\_start}(&args)
\end{align*}
\begin{center}\rule{.85\textwidth}{.01cm}\end{center}
\subsubsection*{(pro) CLEANUP \label{SDIPhaseMapper::cleanup_lab}}
\textbf{Method Documentation:} \\
Cleanup, close any active scans.\newline
\textbf{Arguments:}
\begin{description}
\vspace{-.15cm}
\item[] \hspace{.5cm} \verb"log": No Doc
\end{description}
Example Call:
\begin{align*}
\mathbf{SDIPhaseMapper}-\hspace{-.15cm}>\mathbf{cleanup}, \ &log
\end{align*}
\begin{center}\rule{.85\textwidth}{.01cm}\end{center}
\subsubsection*{(pro) FRAME\_EVENT \label{SDIPhaseMapper::frame_event_lab}}
\textbf{Method Documentation:} \\
Frame event - update the Fourier summations for every pixel, if scan is finished, finalize and unwrap the phasemap, and save it.\newline
\textbf{Arguments:}
\begin{description}
\vspace{-.15cm}
\item[] \hspace{.5cm} \verb"image": Latest frame from the camera
\vspace{-.15cm}
\item[] \hspace{.5cm} \verb"channel": Current scan channel
\end{description}
Example Call:
\begin{align*}
\mathbf{SDIPhaseMapper}-\hspace{-.15cm}>\mathbf{frame\_event}, \ &image,\\ \ &channel
\end{align*}
\begin{center}\rule{.85\textwidth}{.01cm}\end{center}
\subsubsection*{(function) GET\_SETTINGS \label{SDIPhaseMapper::get_settings_lab}}
\textbf{Method Documentation:} \\
Get settings to save.\newline
Takes no arguments \\
Example Call:
\begin{align*}
result = \mathbf{SDIPhaseMapper}-\hspace{-.15cm}>\mathbf{get\_settings}(&)
\end{align*}
\begin{center}\rule{.85\textwidth}{.01cm}\end{center}
\subsubsection*{(pro) SET\_INTERP \label{SDIPhaseMapper::set_interp_lab}}
\textbf{Method Documentation:} \\
When using more than one wavelength to generate a phasemap, we set the order of the cal sources (the numbers corresponding to positions of the calibration source selector switch) and the wavelengths those sources correspond to. The info from both phasemaps is store in such a way as to allow the spectral plugin to interpolate between the phasemaps at the two wavelengths.\newline
\textbf{Arguments:}
\begin{description}
\vspace{-.15cm}
\item[] \hspace{.5cm} \verb"event": Widget event
\end{description}
Example Call:
\begin{align*}
\mathbf{SDIPhaseMapper}-\hspace{-.15cm}>\mathbf{set\_interp}, \ &event
\end{align*}
\begin{center}\rule{.85\textwidth}{.01cm}\end{center}
\subsubsection*{(pro) SET\_NUM\_SCANS \label{SDIPhaseMapper::set_num_scans_lab}}
\textbf{Method Documentation:} \\
Set the number of scans to co-add.\newline
\textbf{Arguments:}
\begin{description}
\vspace{-.15cm}
\item[] \hspace{.5cm} \verb"event": Widget event
\end{description}
Example Call:
\begin{align*}
\mathbf{SDIPhaseMapper}-\hspace{-.15cm}>\mathbf{set\_num\_scans}, \ &event
\end{align*}
\begin{center}\rule{.85\textwidth}{.01cm}\end{center}
\subsubsection*{(pro) SET\_SMOOTH\_WINDOW \label{SDIPhaseMapper::set_smooth_window_lab}}
\textbf{Method Documentation:} \\
Set the width of the smoothing window, applied after phasemap is unwrapped.\newline
\textbf{Arguments:}
\begin{description}
\vspace{-.15cm}
\item[] \hspace{.5cm} \verb"event": Widget event
\end{description}
Example Call:
\begin{align*}
\mathbf{SDIPhaseMapper}-\hspace{-.15cm}>\mathbf{set\_smooth\_window}, \ &event
\end{align*}
\begin{center}\rule{.85\textwidth}{.01cm}\end{center}
\subsubsection*{(pro) START\_SCAN \label{SDIPhaseMapper::start_scan_lab}}
\textbf{Method Documentation:} \\
Start scanning.\newline
\textbf{Arguments:}
\begin{description}
\vspace{-.15cm}
\item[] \hspace{.5cm} \verb"event": Widget event
\end{description}
Example Call:
\begin{align*}
\mathbf{SDIPhaseMapper}-\hspace{-.15cm}>\mathbf{start\_scan}, \ &event
\end{align*}
\begin{center}\rule{.85\textwidth}{.01cm}\end{center}
\subsubsection*{(pro) STOP\_SCAN \label{SDIPhaseMapper::stop_scan_lab}}
\textbf{Method Documentation:} \\
Stop the current scan.\newline
\textbf{Arguments:}
\begin{description}
\vspace{-.15cm}
\item[] \hspace{.5cm} \verb"event": Widget event
\end{description}
Example Call:
\begin{align*}
\mathbf{SDIPhaseMapper}-\hspace{-.15cm}>\mathbf{stop\_scan}, \ &event
\end{align*}
\begin{center}\rule{.85\textwidth}{.01cm}\end{center}
\newpage
\section*{SDISharpness \label{SDISharpness__define_lab}}
\addcontentsline{toc}{section}{SDISharpness}
No Doc\newline \newline
Inherits from: \textbf{XDIBASE} \newline
Class Data: 
\begin{table}[!h]
\begin{tiny}\vspace{-.1cm}\begin{center}
\begin{tabular}{rl|rl|rl}
\hline
(\verb"long") & id& (\verb"float") & sbuffer& (\verb"float") & history \\
(\verb"int") & count& (\verb"int") & bcount& (\verb"float") & best \\
(\verb"int") & leg1\_best& (\verb"int") & leg2\_best& (\verb"int") & leg3\_best \\
(\verb"int") & xcen& (\verb"int") & ycen& (\verb"int") & xdim \\
(\verb"int") & ydim& (\verb"string") & obj\_num& (\verb"structure") & geometry \\
(\verb"int") & need\_frame& (\verb"int") & need\_timer& (\verb"int") & auto \\
(\verb"structure") & palette& (\verb"obj") & manager& (\verb"obj") & console \\
\hline
\end{tabular}\end{center}\end{tiny}\end{table}\vspace{-.5cm} \\
\textbf{Defined in file:} \newline
\small{C:/cal/Operations/SDI\_Instruments/common/idl/core/sdisharpness\_\_define.pro}
\begin{center}\rule{1\textwidth}{.02cm}\end{center}
\subsection*{METHODS:}
\subsubsection*{(function) INIT \label{SDISharpness::init_lab}}
\textbf{Method Documentation:} \\
No Doc\newline
\textbf{Arguments:}
\begin{description}
\vspace{-.15cm}
\item[] \hspace{.5cm} \verb"restore_struc=restore_struc": No Doc
\vspace{-.15cm}
\item[] \hspace{.5cm} \verb"data=data": No Doc
\end{description}
Example Call:
\begin{align*}
result = \mathbf{SDISharpness}-\hspace{-.15cm}>\mathbf{init}(&restore_struc=restore_struc,\\ \ &data=data)
\end{align*}
\begin{center}\rule{.85\textwidth}{.01cm}\end{center}
\subsubsection*{(pro) CLEANUP \label{SDISharpness::cleanup_lab}}
\textbf{Method Documentation:} \\
No Doc\newline
\textbf{Arguments:}
\begin{description}
\vspace{-.15cm}
\item[] \hspace{.5cm} \verb"log": No Doc
\end{description}
Example Call:
\begin{align*}
\mathbf{SDISharpness}-\hspace{-.15cm}>\mathbf{cleanup}, \ &log
\end{align*}
\begin{center}\rule{.85\textwidth}{.01cm}\end{center}
\subsubsection*{(pro) FRAME\_EVENT \label{SDISharpness::frame_event_lab}}
\textbf{Method Documentation:} \\
No Doc\newline
\textbf{Arguments:}
\begin{description}
\vspace{-.15cm}
\item[] \hspace{.5cm} \verb"image": No Doc
\vspace{-.15cm}
\item[] \hspace{.5cm} \verb"channel": No Doc
\vspace{-.15cm}
\item[] \hspace{.5cm} \verb"scan": No Doc
\end{description}
Example Call:
\begin{align*}
\mathbf{SDISharpness}-\hspace{-.15cm}>\mathbf{frame\_event}, \ &image,\\ \ &channel,\\ \ &scan
\end{align*}
\begin{center}\rule{.85\textwidth}{.01cm}\end{center}
\subsubsection*{(pro) GET\_CENTER \label{SDISharpness::get_center_lab}}
\textbf{Method Documentation:} \\
No Doc\newline
\textbf{Arguments:}
\begin{description}
\vspace{-.15cm}
\item[] \hspace{.5cm} \verb"event": No Doc
\end{description}
Example Call:
\begin{align*}
\mathbf{SDISharpness}-\hspace{-.15cm}>\mathbf{get\_center}, \ &event
\end{align*}
\begin{center}\rule{.85\textwidth}{.01cm}\end{center}
\subsubsection*{(function) GET\_SETTINGS \label{SDISharpness::get_settings_lab}}
\textbf{Method Documentation:} \\
No Doc\newline
Takes no arguments \\
Example Call:
\begin{align*}
result = \mathbf{SDISharpness}-\hspace{-.15cm}>\mathbf{get\_settings}(&)
\end{align*}
\begin{center}\rule{.85\textwidth}{.01cm}\end{center}
\newpage
\section*{SDISpectrum \label{SDISpectrum__define_lab}}
\addcontentsline{toc}{section}{SDISpectrum}
The Spectrum plugin acquires spectra and saves them to netcdf files. It requires a wavelength, a filename to save to and a zonemap file to use for dividing up the field of view. Some things are hard coded in this which should probably be made configurable (see the frame\_event method). After each complete exposure, this plugin will attempt to send back a snapshot of the most recent acquisition to an ftp server for real-time data processing (the snapshot is sent back if the logging.ftp\_snapshot field is populated in the settings file, see XDIConsole::spectrum\_snapshot for the details of this).\newline \newline
Inherits from: \textbf{XDIBASE} \newline
Class Data: 
\begin{table}[!h]
\begin{tiny}\vspace{-.1cm}\begin{center}
\begin{tabular}{rl|rl|rl}
\hline
(\verb"long") & id& (\verb"int") & scanning& (\verb"int") & nchann \\
(\verb"int") & xdim& (\verb"int") & ydim& (\verb"int") & save\_file\_id \\
(\verb"ptr") & spectra& (\verb"ptr") & last\_spectra& (\verb"ptr") & zonemap \\
(\verb"ptr") & zonemap\_boundaries& (\verb"ptr") & phasemap& (\verb"float") & signal\_noise\_history \\
(\verb"float") & channel\_background\_history& (\verb"float") & scan\_background\_history& (\verb"ptr") & zone\_centers \\
(\verb"int") & nzones& (\verb"string") & dll& (\verb"int") & nscans \\
(\verb"int") & file\_id& (\verb"string") & zone\_settings& (\verb"float") & wavelength \\
(\verb"float") & a& (\verb"float") & b& (\verb"float") & c \\
(\verb"double") & scan\_start\_time& (\verb"string") & spec\_path& (\verb"int") & nrings \\
(\verb"string") & file\_name\_format& (\verb"string") & filename& (\verb"ptr") & rads \\
(\verb"ptr") & secs& (\verb"ptr") & accumulated\_image& (\verb"int") & finalize\_flag \\
(\verb"string") & insprof\_filename& (\verb"float") & etalon\_gap& (\verb"string") & obj\_num \\
(\verb"structure") & geometry& (\verb"int") & need\_frame& (\verb"int") & need\_timer \\
(\verb"int") & auto& (\verb"structure") & palette& (\verb"obj") & manager \\
\hline
\end{tabular}\end{center}\end{tiny}\end{table}\vspace{-.5cm} \\
\textbf{Defined in file:} \newline
\small{C:/cal/Operations/SDI\_Instruments/common/idl/core/sdispectrum\_\_define.pro}
\begin{center}\rule{1\textwidth}{.02cm}\end{center}
\subsection*{METHODS:}
\subsubsection*{(function) INIT \label{SDISpectrum::init_lab}}
\textbf{Method Documentation:} \\
Spectrum initializer, make sure we have a filename format and a zone map.\newline
\textbf{Arguments:}
\begin{description}
\vspace{-.15cm}
\item[] \hspace{.5cm} \verb"restore_struc=restore_struc": Restored settings
\vspace{-.15cm}
\item[] \hspace{.5cm} \verb"data=data": Misc data from the console
\vspace{-.15cm}
\item[] \hspace{.5cm} \verb"zone_settings=zone_settings": A zone settings file name
\vspace{-.15cm}
\item[] \hspace{.5cm} \verb"file_name_format=file_name_format": Format for generating the netcdf file name
\end{description}
Example Call:
\begin{align*}
result = \mathbf{SDISpectrum}-\hspace{-.15cm}>\mathbf{init}(&restore_struc=restore_struc,\\ \ &data=data,\\ \ &zone_settings=zone_settings,\\ \ &file_name_format=file_name_format)
\end{align*}
\begin{center}\rule{.85\textwidth}{.01cm}\end{center}
\subsubsection*{(function) AUTO\_START \label{SDISpectrum::auto_start_lab}}
\textbf{Method Documentation:} \\
Auto-start method, called when running in auto-mode.\newline
\textbf{Arguments:}
\begin{description}
\vspace{-.15cm}
\item[] \hspace{.5cm} \verb"args": String array of arguments from the schedule file
\end{description}
Example Call:
\begin{align*}
result = \mathbf{SDISpectrum}-\hspace{-.15cm}>\mathbf{auto\_start}(&args)
\end{align*}
\begin{center}\rule{.85\textwidth}{.01cm}\end{center}
\subsubsection*{(pro) CLEANUP \label{SDISpectrum::cleanup_lab}}
\textbf{Method Documentation:} \\
Cleanup - stop any active scans, close the netcdf file, free pointers.\newline
\textbf{Arguments:}
\begin{description}
\vspace{-.15cm}
\item[] \hspace{.5cm} \verb"log": No Doc
\end{description}
Example Call:
\begin{align*}
\mathbf{SDISpectrum}-\hspace{-.15cm}>\mathbf{cleanup}, \ &log
\end{align*}
\begin{center}\rule{.85\textwidth}{.01cm}\end{center}
\subsubsection*{(pro) FINALIZE\_SCAN \label{SDISpectrum::finalize_scan_lab}}
\textbf{Method Documentation:} \\
Called when a user clicks on the "Finalize" button, to indicate that an exposure should be finished after the next scan, regardless of signal-to-noise, etc.\newline
\textbf{Arguments:}
\begin{description}
\vspace{-.15cm}
\item[] \hspace{.5cm} \verb"event": Widget event
\end{description}
Example Call:
\begin{align*}
\mathbf{SDISpectrum}-\hspace{-.15cm}>\mathbf{finalize\_scan}, \ &event
\end{align*}
\begin{center}\rule{.85\textwidth}{.01cm}\end{center}
\subsubsection*{(pro) FIT\_SPECTRA \label{SDISpectrum::fit_spectra_lab}}
\textbf{Method Documentation:} \\
Fit spectra and create skymaps of peak position and temperature, and display them. This function was introduced to diagnose Mawson camera problems, and has stuck around since it may be generally useful.\newline
\textbf{Arguments:}
\begin{description}
\vspace{-.15cm}
\item[] \hspace{.5cm} \verb"event": Widget event
\end{description}
Example Call:
\begin{align*}
\mathbf{SDISpectrum}-\hspace{-.15cm}>\mathbf{fit\_spectra}, \ &event
\end{align*}
\begin{center}\rule{.85\textwidth}{.01cm}\end{center}
\subsubsection*{(pro) FRAME\_EVENT \label{SDISpectrum::frame_event_lab}}
\textbf{Method Documentation:} \\
Frame event where the spectral information from the latest camera image is extracted. The primary purpose of this function is to call "uUpdateSpectra" in the SDI\_External dll, which updates the current spectral information based on the latest camera frame. This function also checks to see if exposures are finished, sends real-time data snapshots to the console for ftp-ing, accumulates the background `allsky' image, and updates the display of spectra and signal/noise history.\newline
\textbf{Arguments:}
\begin{description}
\vspace{-.15cm}
\item[] \hspace{.5cm} \verb"image": Latest camera image
\vspace{-.15cm}
\item[] \hspace{.5cm} \verb"channel": Current scan channel
\end{description}
Example Call:
\begin{align*}
\mathbf{SDISpectrum}-\hspace{-.15cm}>\mathbf{frame\_event}, \ &image,\\ \ &channel
\end{align*}
\begin{center}\rule{.85\textwidth}{.01cm}\end{center}
\subsubsection*{(function) GET\_SETTINGS \label{SDISpectrum::get_settings_lab}}
\textbf{Method Documentation:} \\
Get settings to save.\newline
Takes no arguments \\
Example Call:
\begin{align*}
result = \mathbf{SDISpectrum}-\hspace{-.15cm}>\mathbf{get\_settings}(&)
\end{align*}
\begin{center}\rule{.85\textwidth}{.01cm}\end{center}
\subsubsection*{(pro) INITIALIZER \label{SDISpectrum::initializer_lab}}
\textbf{Method Documentation:} \\
Initialize plugin variables, prepare the phase map, create a zone map.\newline
Takes no arguments \\
Example Call:
\begin{align*}
\mathbf{SDISpectrum}-\hspace{-.15cm}>\mathbf{initializer}
\end{align*}
\begin{center}\rule{.85\textwidth}{.01cm}\end{center}
\subsubsection*{(pro) SET\_PHASEMAP \label{SDISpectrum::set_phasemap_lab}}
\textbf{Method Documentation:} \\
Set-up the phasemap, that is, retrieve phase map parameters from the console, interpolate to the spectrum wavelength, and wrap the phase map.\newline
\textbf{Arguments:}
\begin{description}
\vspace{-.15cm}
\item[] \hspace{.5cm} \verb"failed": OUT: flag to indicate failure, not currently used (returns 0)
\end{description}
Example Call:
\begin{align*}
\mathbf{SDISpectrum}-\hspace{-.15cm}>\mathbf{set\_phasemap}, \ &failed
\end{align*}
\begin{center}\rule{.85\textwidth}{.01cm}\end{center}
\subsubsection*{(pro) START\_SCAN \label{SDISpectrum::start_scan_lab}}
\textbf{Method Documentation:} \\
Start scanning.\newline
\textbf{Arguments:}
\begin{description}
\vspace{-.15cm}
\item[] \hspace{.5cm} \verb"event": Widget event
\end{description}
Example Call:
\begin{align*}
\mathbf{SDISpectrum}-\hspace{-.15cm}>\mathbf{start\_scan}, \ &event
\end{align*}
\begin{center}\rule{.85\textwidth}{.01cm}\end{center}
\subsubsection*{(pro) STOP\_SCAN \label{SDISpectrum::stop_scan_lab}}
\textbf{Method Documentation:} \\
Stop a currently active scan.\newline
\textbf{Arguments:}
\begin{description}
\vspace{-.15cm}
\item[] \hspace{.5cm} \verb"event": Widget event
\end{description}
Example Call:
\begin{align*}
\mathbf{SDISpectrum}-\hspace{-.15cm}>\mathbf{stop\_scan}, \ &event
\end{align*}
\begin{center}\rule{.85\textwidth}{.01cm}\end{center}
\newpage
\section*{SDIStepsPerOrder \label{SDIStepsPerOrder__define_lab}}
\addcontentsline{toc}{section}{SDIStepsPerOrder}
The StepsPerOrder plugin is used to calculate the size of the `voltage' increment that needs to be applied to each etalon leg at each channel in a scan such that a full scan corresponds to a unit change in interference order.\newline \newline
Inherits from: \textbf{XDIBASE} \newline
Class Data: 
\begin{table}[!h]
\begin{tiny}\vspace{-.1cm}\begin{center}
\begin{tabular}{rl|rl|rl}
\hline
(\verb"long") & id& (\verb"ptr") & corr& (\verb"int") & num\_chords \\
(\verb"int") & curr\_chord& (\verb"int") & scanning& (\verb"int") & nchann \\
(\verb"int") & start\_volt\_offset& (\verb"int") & stop\_volt\_offset& (\verb"float") & volt\_step\_size \\
(\verb"obj") & scan\_obj& (\verb"int") & curr\_chann& (\verb"int") & last\_chann \\
(\verb"ptr") & image& (\verb"ptr") & ref\_image& (\verb"int") & xdim \\
(\verb"int") & ydim& (\verb"int") & counter& (\verb"int") & last\_counter \\
(\verb"ptr") & chord\_hist& (\verb"float") & wavelength& (\verb"int") & record\_value \\
(\verb"string") & record\_file& (\verb"float") & gain& (\verb"float") & exptime \\
(\verb"string") & obj\_num& (\verb"structure") & geometry& (\verb"int") & need\_frame \\
(\verb"int") & need\_timer& (\verb"int") & auto& (\verb"structure") & palette \\
\hline
\end{tabular}\end{center}\end{tiny}\end{table}\vspace{-.5cm} \\
\textbf{Defined in file:} \newline
\small{C:/cal/Operations/SDI\_Instruments/common/idl/core/sdistepsperorder\_\_define.pro}
\begin{center}\rule{1\textwidth}{.02cm}\end{center}
\subsection*{METHODS:}
\subsubsection*{(function) INIT \label{SDIStepsPerOrder::init_lab}}
\textbf{Method Documentation:} \\
Initialize the StepsPerOrder plugin.\newline
\textbf{Arguments:}
\begin{description}
\vspace{-.15cm}
\item[] \hspace{.5cm} \verb"restore_struc=restore_struc": Restored settings
\vspace{-.15cm}
\item[] \hspace{.5cm} \verb"data=data": Misc data from the console
\end{description}
Example Call:
\begin{align*}
result = \mathbf{SDIStepsPerOrder}-\hspace{-.15cm}>\mathbf{init}(&restore_struc=restore_struc,\\ \ &data=data)
\end{align*}
\begin{center}\rule{.85\textwidth}{.01cm}\end{center}
\subsubsection*{(function) AUTO\_START \label{SDIStepsPerOrder::auto_start_lab}}
\textbf{Method Documentation:} \\
Auto-start called when running in auto-mode.\newline
\textbf{Arguments:}
\begin{description}
\vspace{-.15cm}
\item[] \hspace{.5cm} \verb"args": String array of arguments from the schedule file
\end{description}
Example Call:
\begin{align*}
result = \mathbf{SDIStepsPerOrder}-\hspace{-.15cm}>\mathbf{auto\_start}(&args)
\end{align*}
\begin{center}\rule{.85\textwidth}{.01cm}\end{center}
\subsubsection*{(pro) CLEANUP \label{SDIStepsPerOrder::cleanup_lab}}
\textbf{Method Documentation:} \\
Cleanup - free pointers, stop any active scan.\newline
\textbf{Arguments:}
\begin{description}
\vspace{-.15cm}
\item[] \hspace{.5cm} \verb"log": No Doc
\end{description}
Example Call:
\begin{align*}
\mathbf{SDIStepsPerOrder}-\hspace{-.15cm}>\mathbf{cleanup}, \ &log
\end{align*}
\begin{center}\rule{.85\textwidth}{.01cm}\end{center}
\subsubsection*{(pro) FRAME\_EVENT \label{SDIStepsPerOrder::frame_event_lab}}
\textbf{Method Documentation:} \\
Process the latest camera frame: bascially calculate the correlation between the current camera image and a reference image, store this value in a vector. If finished scanning, fit the vector of correlation values to find the peak, and calculate the steps/order value based on the position of that peak and the number of channels in a scan.\newline
\textbf{Arguments:}
\begin{description}
\vspace{-.15cm}
\item[] \hspace{.5cm} \verb"image": Latest camera image
\vspace{-.15cm}
\item[] \hspace{.5cm} \verb"channel": Current scan channel
\end{description}
Example Call:
\begin{align*}
\mathbf{SDIStepsPerOrder}-\hspace{-.15cm}>\mathbf{frame\_event}, \ &image,\\ \ &channel
\end{align*}
\begin{center}\rule{.85\textwidth}{.01cm}\end{center}
\subsubsection*{(function) GET\_SETTINGS \label{SDIStepsPerOrder::get_settings_lab}}
\textbf{Method Documentation:} \\
Get settings to save.\newline
Takes no arguments \\
Example Call:
\begin{align*}
result = \mathbf{SDIStepsPerOrder}-\hspace{-.15cm}>\mathbf{get\_settings}(&)
\end{align*}
\begin{center}\rule{.85\textwidth}{.01cm}\end{center}
\subsubsection*{(pro) START\_SCAN \label{SDIStepsPerOrder::start_scan_lab}}
\textbf{Method Documentation:} \\
Start scanning, set-up variables.\newline
\textbf{Arguments:}
\begin{description}
\vspace{-.15cm}
\item[] \hspace{.5cm} \verb"event": Widget event
\end{description}
Example Call:
\begin{align*}
\mathbf{SDIStepsPerOrder}-\hspace{-.15cm}>\mathbf{start\_scan}, \ &event
\end{align*}
\begin{center}\rule{.85\textwidth}{.01cm}\end{center}
\subsubsection*{(pro) STOP\_SCAN \label{SDIStepsPerOrder::stop_scan_lab}}
\textbf{Method Documentation:} \\
Stop the current scan, no steps/order value will be saved.\newline
\textbf{Arguments:}
\begin{description}
\vspace{-.15cm}
\item[] \hspace{.5cm} \verb"event": Widget event
\end{description}
Example Call:
\begin{align*}
\mathbf{SDIStepsPerOrder}-\hspace{-.15cm}>\mathbf{stop\_scan}, \ &event
\end{align*}
\begin{center}\rule{.85\textwidth}{.01cm}\end{center}
\subsubsection*{(pro) TOGGLE\_RECORD \label{SDIStepsPerOrder::Toggle_Record_lab}}
\textbf{Method Documentation:} \\
Toggle on/off the option to record steps/order values to a dedicated log file. This option is located under the file menu of the plugin, and will be rememberd for this plugin.\newline
\textbf{Arguments:}
\begin{description}
\vspace{-.15cm}
\item[] \hspace{.5cm} \verb"event": Widget event
\end{description}
Example Call:
\begin{align*}
\mathbf{SDIStepsPerOrder}-\hspace{-.15cm}>\mathbf{Toggle\_Record}, \ &event
\end{align*}
\begin{center}\rule{.85\textwidth}{.01cm}\end{center}
\newpage
\section*{SDIVidshow \label{SDIVidshow__define_lab}}
\addcontentsline{toc}{section}{SDIVidshow}
The Vidshow plugin displays the latest camera images as they are recorded.\newline \newline
Inherits from: \textbf{XDIBASE} \newline
Class Data: 
\begin{table}[!h]
\begin{tiny}\vspace{-.1cm}\begin{center}
\begin{tabular}{rl|rl|rl}
\hline
(\verb"long") & id& (\verb"int") & inst& (\verb"float") & exp\_time \\
(\verb"int") & xdim& (\verb"int") & ydim& (\verb"int") & scale \\
(\verb"float") & scale\_fac& (\verb"int") & crosshairs& (\verb"int") & crosshairs\_point \\
(\verb"int") & grid& (\verb"int") & color\_table& (\verb"long") & framecount \\
(\verb"double") & tstrt& (\verb"int") & mask\_quadrants& (\verb"string") & obj\_num \\
(\verb"structure") & geometry& (\verb"int") & need\_frame& (\verb"int") & need\_timer \\
(\verb"int") & auto& (\verb"structure") & palette& (\verb"obj") & manager \\
\hline
\end{tabular}\end{center}\end{tiny}\end{table}\vspace{-.5cm} \\
\textbf{Defined in file:} \newline
\small{C:/cal/Operations/SDI\_Instruments/common/idl/core/sdividshow\_\_define.pro}
\begin{center}\rule{1\textwidth}{.02cm}\end{center}
\subsection*{METHODS:}
\subsubsection*{(function) INIT \label{SDIVidshow::init_lab}}
\textbf{Method Documentation:} \\
Initialize the Vidshow plugin.\newline
\textbf{Arguments:}
\begin{description}
\vspace{-.15cm}
\item[] \hspace{.5cm} \verb"restore_struc=restore_struc": Restored settings
\vspace{-.15cm}
\item[] \hspace{.5cm} \verb"data=data": Misc data from the console
\end{description}
Example Call:
\begin{align*}
result = \mathbf{SDIVidshow}-\hspace{-.15cm}>\mathbf{init}(&restore_struc=restore_struc,\\ \ &data=data)
\end{align*}
\begin{center}\rule{.85\textwidth}{.01cm}\end{center}
\subsubsection*{(pro) CLEANUP \label{SDIVidshow::cleanup_lab}}
\textbf{Method Documentation:} \\
Cleanup - nothing to do.\newline
\textbf{Arguments:}
\begin{description}
\vspace{-.15cm}
\item[] \hspace{.5cm} \verb"log": No Doc
\end{description}
Example Call:
\begin{align*}
\mathbf{SDIVidshow}-\hspace{-.15cm}>\mathbf{cleanup}, \ &log
\end{align*}
\begin{center}\rule{.85\textwidth}{.01cm}\end{center}
\subsubsection*{(pro) FIT\_WINDOW \label{SDIVidshow::fit_window_lab}}
\textbf{Method Documentation:} \\
Resize the window to fit the native resolution of the camera image, called from the menu.\newline
\textbf{Arguments:}
\begin{description}
\vspace{-.15cm}
\item[] \hspace{.5cm} \verb"event": Widget event
\end{description}
Example Call:
\begin{align*}
\mathbf{SDIVidshow}-\hspace{-.15cm}>\mathbf{fit\_window}, \ &event
\end{align*}
\begin{center}\rule{.85\textwidth}{.01cm}\end{center}
\subsubsection*{(pro) FRAME\_EVENT \label{SDIVidshow::frame_event_lab}}
\textbf{Method Documentation:} \\
Receive a new camera frame, scale it and display.\newline
\textbf{Arguments:}
\begin{description}
\vspace{-.15cm}
\item[] \hspace{.5cm} \verb"image": Latest camera image
\vspace{-.15cm}
\item[] \hspace{.5cm} \verb"channel": Current scan channel
\end{description}
Example Call:
\begin{align*}
\mathbf{SDIVidshow}-\hspace{-.15cm}>\mathbf{frame\_event}, \ &image,\\ \ &channel
\end{align*}
\begin{center}\rule{.85\textwidth}{.01cm}\end{center}
\subsubsection*{(function) GET\_SETTINGS \label{SDIVidshow::get_settings_lab}}
\textbf{Method Documentation:} \\
Get settings to save.\newline
Takes no arguments \\
Example Call:
\begin{align*}
result = \mathbf{SDIVidshow}-\hspace{-.15cm}>\mathbf{get\_settings}(&)
\end{align*}
\begin{center}\rule{.85\textwidth}{.01cm}\end{center}
\subsubsection*{(pro) MASK\_QUADRANTS \label{SDIVidshow::mask_quadrants_lab}}
\textbf{Method Documentation:} \\
Mask out most of the four quadrants of the image, leaving only a small `cross' of the image left to display, helps for slow connections, called from the menu.\newline
\textbf{Arguments:}
\begin{description}
\vspace{-.15cm}
\item[] \hspace{.5cm} \verb"event": Widget event
\end{description}
Example Call:
\begin{align*}
\mathbf{SDIVidshow}-\hspace{-.15cm}>\mathbf{mask\_quadrants}, \ &event
\end{align*}
\begin{center}\rule{.85\textwidth}{.01cm}\end{center}
\subsubsection*{(pro) SCALING \label{SDIVidshow::scaling_lab}}
\textbf{Method Documentation:} \\
Toggle between using the manual scale factor and auto scaling, called from the menu.\newline
\textbf{Arguments:}
\begin{description}
\vspace{-.15cm}
\item[] \hspace{.5cm} \verb"event": Widget event
\end{description}
Example Call:
\begin{align*}
\mathbf{SDIVidshow}-\hspace{-.15cm}>\mathbf{scaling}, \ &event
\end{align*}
\begin{center}\rule{.85\textwidth}{.01cm}\end{center}
\subsubsection*{(pro) SET\_COLOR\_TABLE \label{SDIVidshow::set_color_table_lab}}
\textbf{Method Documentation:} \\
Set the color table, called when user selects this option from the menu.\newline
\textbf{Arguments:}
\begin{description}
\vspace{-.15cm}
\item[] \hspace{.5cm} \verb"event": Widget event
\end{description}
Example Call:
\begin{align*}
\mathbf{SDIVidshow}-\hspace{-.15cm}>\mathbf{set\_color\_table}, \ &event
\end{align*}
\begin{center}\rule{.85\textwidth}{.01cm}\end{center}
\subsubsection*{(pro) SET\_CROSSHAIRS \label{SDIVidshow::set_crosshairs_lab}}
\textbf{Method Documentation:} \\
Toggle on/off diaplying the crosshairs, called from the menu.\newline
\textbf{Arguments:}
\begin{description}
\vspace{-.15cm}
\item[] \hspace{.5cm} \verb"event": Widget event
\end{description}
Example Call:
\begin{align*}
\mathbf{SDIVidshow}-\hspace{-.15cm}>\mathbf{set\_crosshairs}, \ &event
\end{align*}
\begin{center}\rule{.85\textwidth}{.01cm}\end{center}
\subsubsection*{(pro) SET\_CROSSHAIRS\_POINT \label{SDIVidshow::set_crosshairs_point_lab}}
\textbf{Method Documentation:} \\
Set where the crosshairs intersect (x, y), called from the menu.\newline
\textbf{Arguments:}
\begin{description}
\vspace{-.15cm}
\item[] \hspace{.5cm} \verb"event": Widget event
\end{description}
Example Call:
\begin{align*}
\mathbf{SDIVidshow}-\hspace{-.15cm}>\mathbf{set\_crosshairs\_point}, \ &event
\end{align*}
\begin{center}\rule{.85\textwidth}{.01cm}\end{center}
\subsubsection*{(pro) SET\_GRID \label{SDIVidshow::set_grid_lab}}
\textbf{Method Documentation:} \\
Toggle on/off displaying a grid overlay, called from the menu.\newline
\textbf{Arguments:}
\begin{description}
\vspace{-.15cm}
\item[] \hspace{.5cm} \verb"event": Widget event
\end{description}
Example Call:
\begin{align*}
\mathbf{SDIVidshow}-\hspace{-.15cm}>\mathbf{set\_grid}, \ &event
\end{align*}
\begin{center}\rule{.85\textwidth}{.01cm}\end{center}
\subsubsection*{(pro) SET\_SCALE \label{SDIVidshow::set_scale_lab}}
\textbf{Method Documentation:} \\
Set a manual scale value applied to image prior to display, called from the menu.\newline
\textbf{Arguments:}
\begin{description}
\vspace{-.15cm}
\item[] \hspace{.5cm} \verb"event": Widget event
\end{description}
Example Call:
\begin{align*}
\mathbf{SDIVidshow}-\hspace{-.15cm}>\mathbf{set\_scale}, \ &event
\end{align*}
\begin{center}\rule{.85\textwidth}{.01cm}\end{center}
\newpage
\section*{XDIBase \label{XDIBase__define_lab}}
\addcontentsline{toc}{section}{XDIBase}
This class defined basic properties all plugins inherit, like geometry, references to the console and widget manager, flags like need\_timer and need\_frame, etc. For a plugin to work, it must inherit from XDIBase.\newline \newline
Inherits from: \textbf{None} \newline
Class Data: 
\begin{table}[!h]
\begin{tiny}\vspace{-.1cm}\begin{center}
\begin{tabular}{rl|rl|rl}
\hline
(\verb"string") & obj\_num& (\verb"structure") & geometry& (\verb"int") & need\_frame \\
(\verb"int") & need\_timer& (\verb"int") & auto& (\verb"structure") & palette \\
\hline
\end{tabular}\end{center}\end{tiny}\end{table}\vspace{-.5cm} \\
\textbf{Defined in file:} \newline
\small{C:/cal/Operations/SDI\_Instruments/common/idl/core/xdibase\_\_define.pro}
\begin{center}\rule{1\textwidth}{.02cm}\end{center}
\newpage
\section*{XDIConsole \label{XDIConsole__define_lab}}
\addcontentsline{toc}{section}{XDIConsole}
No Doc\newline \newline
Inherits from: \textbf{XDIBASE} \newline
Class Data: 
\begin{table}[!h]
\begin{tiny}\vspace{-.1cm}\begin{center}
\begin{tabular}{rl|rl|rl}
\hline
(\verb"structure") & etalon& (\verb"structure") & camera& (\verb"structure") & header \\
(\verb"structure") & logging& (\verb"structure") & misc& (\verb"structure") & runtime \\
(\verb"structure") & buffer& (\verb"string") & obj\_num& (\verb"structure") & geometry \\
(\verb"int") & need\_frame& (\verb"int") & need\_timer& (\verb"int") & auto \\
(\verb"structure") & palette& (\verb"obj") & manager& (\verb"obj") & console \\
\hline
\end{tabular}\end{center}\end{tiny}\end{table}\vspace{-.5cm} \\
\textbf{Defined in file:} \newline
\small{C:/cal/Operations/SDI\_Instruments/common/idl/core/xdiconsole\_\_define.pro}
\begin{center}\rule{1\textwidth}{.02cm}\end{center}
\subsection*{METHODS:}
\subsubsection*{(function) INIT \label{XDIConsole::init_lab}}
\textbf{Method Documentation:} \\
No Doc\newline
\textbf{Arguments:}
\begin{description}
\vspace{-.15cm}
\item[] \hspace{.5cm} \verb"schedule=schedule": No Doc
\vspace{-.15cm}
\item[] \hspace{.5cm} \verb"mode=mode": No Doc
\vspace{-.15cm}
\item[] \hspace{.5cm} \verb"settings=settings": No Doc
\vspace{-.15cm}
\item[] \hspace{.5cm} \verb"start_line=start_line": No Doc
\end{description}
Example Call:
\begin{align*}
result = \mathbf{XDIConsole}-\hspace{-.15cm}>\mathbf{init}(&schedule=schedule,\\ \ &mode=mode,\\ \ &settings=settings,\\ \ &start_line=start_line)
\end{align*}
\begin{center}\rule{.85\textwidth}{.01cm}\end{center}
\subsubsection*{(pro) CAM\_COOLER \label{XDIConsole::cam_cooler_lab}}
\textbf{Method Documentation:} \\
No Doc\newline
\textbf{Arguments:}
\begin{description}
\vspace{-.15cm}
\item[] \hspace{.5cm} \verb"event": No Doc
\end{description}
Example Call:
\begin{align*}
\mathbf{XDIConsole}-\hspace{-.15cm}>\mathbf{cam\_cooler}, \ &event
\end{align*}
\begin{center}\rule{.85\textwidth}{.01cm}\end{center}
\subsubsection*{(pro) CAM\_COOLER\_EVENT \label{XDIConsole::cam_cooler_event_lab}}
\textbf{Method Documentation:} \\
No Doc\newline
\textbf{Arguments:}
\begin{description}
\vspace{-.15cm}
\item[] \hspace{.5cm} \verb"event": No Doc
\end{description}
Example Call:
\begin{align*}
\mathbf{XDIConsole}-\hspace{-.15cm}>\mathbf{cam\_cooler\_event}, \ &event
\end{align*}
\begin{center}\rule{.85\textwidth}{.01cm}\end{center}
\subsubsection*{(pro) CAM\_EXPTIME \label{XDIConsole::cam_exptime_lab}}
\textbf{Method Documentation:} \\
No Doc\newline
\textbf{Arguments:}
\begin{description}
\vspace{-.15cm}
\item[] \hspace{.5cm} \verb"event": No Doc
\vspace{-.15cm}
\item[] \hspace{.5cm} \verb"new_time=new_time": No Doc
\end{description}
Example Call:
\begin{align*}
\mathbf{XDIConsole}-\hspace{-.15cm}>\mathbf{cam\_exptime}, \ &event,\\ \ &new_time=new_time
\end{align*}
\begin{center}\rule{.85\textwidth}{.01cm}\end{center}
\subsubsection*{(pro) CAM\_GAIN \label{XDIConsole::cam_gain_lab}}
\textbf{Method Documentation:} \\
No Doc\newline
\textbf{Arguments:}
\begin{description}
\vspace{-.15cm}
\item[] \hspace{.5cm} \verb"event": No Doc
\vspace{-.15cm}
\item[] \hspace{.5cm} \verb"new_gain=new_gain": No Doc
\end{description}
Example Call:
\begin{align*}
\mathbf{XDIConsole}-\hspace{-.15cm}>\mathbf{cam\_gain}, \ &event,\\ \ &new_gain=new_gain
\end{align*}
\begin{center}\rule{.85\textwidth}{.01cm}\end{center}
\subsubsection*{(pro) CAM\_INITIALIZE \label{XDIConsole::cam_initialize_lab}}
\textbf{Method Documentation:} \\
No Doc\newline
\textbf{Arguments:}
\begin{description}
\vspace{-.15cm}
\item[] \hspace{.5cm} \verb"event": No Doc
\end{description}
Example Call:
\begin{align*}
\mathbf{XDIConsole}-\hspace{-.15cm}>\mathbf{cam\_initialize}, \ &event
\end{align*}
\begin{center}\rule{.85\textwidth}{.01cm}\end{center}
\subsubsection*{(pro) CAM\_SHUTDOWN \label{XDIConsole::cam_shutdown_lab}}
\textbf{Method Documentation:} \\
No Doc\newline
\textbf{Arguments:}
\begin{description}
\vspace{-.15cm}
\item[] \hspace{.5cm} \verb"event": No Doc
\end{description}
Example Call:
\begin{align*}
\mathbf{XDIConsole}-\hspace{-.15cm}>\mathbf{cam\_shutdown}, \ &event
\end{align*}
\begin{center}\rule{.85\textwidth}{.01cm}\end{center}
\subsubsection*{(pro) CAM\_SHUTTERCLOSE \label{XDIConsole::cam_shutterclose_lab}}
\textbf{Method Documentation:} \\
No Doc\newline
\textbf{Arguments:}
\begin{description}
\vspace{-.15cm}
\item[] \hspace{.5cm} \verb"event": No Doc
\vspace{-.15cm}
\item[] \hspace{.5cm} \verb"shutdown=shutdown": No Doc
\end{description}
Example Call:
\begin{align*}
\mathbf{XDIConsole}-\hspace{-.15cm}>\mathbf{cam\_shutterclose}, \ &event,\\ \ &shutdown=shutdown
\end{align*}
\begin{center}\rule{.85\textwidth}{.01cm}\end{center}
\subsubsection*{(pro) CAM\_SHUTTEROPEN \label{XDIConsole::cam_shutteropen_lab}}
\textbf{Method Documentation:} \\
No Doc\newline
\textbf{Arguments:}
\begin{description}
\vspace{-.15cm}
\item[] \hspace{.5cm} \verb"event": No Doc
\end{description}
Example Call:
\begin{align*}
\mathbf{XDIConsole}-\hspace{-.15cm}>\mathbf{cam\_shutteropen}, \ &event
\end{align*}
\begin{center}\rule{.85\textwidth}{.01cm}\end{center}
\subsubsection*{(pro) CAM\_STATUS \label{XDIConsole::cam_status_lab}}
\textbf{Method Documentation:} \\
No Doc\newline
\textbf{Arguments:}
\begin{description}
\vspace{-.15cm}
\item[] \hspace{.5cm} \verb"event": No Doc
\end{description}
Example Call:
\begin{align*}
\mathbf{XDIConsole}-\hspace{-.15cm}>\mathbf{cam\_status}, \ &event
\end{align*}
\begin{center}\rule{.85\textwidth}{.01cm}\end{center}
\subsubsection*{(pro) CAM\_TEMP \label{XDIConsole::cam_temp_lab}}
\textbf{Method Documentation:} \\
No Doc\newline
\textbf{Arguments:}
\begin{description}
\vspace{-.15cm}
\item[] \hspace{.5cm} \verb"event": No Doc
\end{description}
Example Call:
\begin{align*}
\mathbf{XDIConsole}-\hspace{-.15cm}>\mathbf{cam\_temp}, \ &event
\end{align*}
\begin{center}\rule{.85\textwidth}{.01cm}\end{center}
\subsubsection*{(pro) CLEANUP \label{XDIConsole::cleanup_lab}}
\textbf{Method Documentation:} \\
No Doc\newline
Takes no arguments \\
Example Call:
\begin{align*}
\mathbf{XDIConsole}-\hspace{-.15cm}>\mathbf{cleanup}
\end{align*}
\begin{center}\rule{.85\textwidth}{.01cm}\end{center}
\subsubsection*{(pro) CLOSE\_MPORT \label{XDIConsole::close_mport_lab}}
\textbf{Method Documentation:} \\
No Doc\newline
\textbf{Arguments:}
\begin{description}
\vspace{-.15cm}
\item[] \hspace{.5cm} \verb"event": No Doc
\end{description}
Example Call:
\begin{align*}
\mathbf{XDIConsole}-\hspace{-.15cm}>\mathbf{close\_mport}, \ &event
\end{align*}
\begin{center}\rule{.85\textwidth}{.01cm}\end{center}
\subsubsection*{(pro) EDIT\_PORTS \label{XDIConsole::edit_ports_lab}}
\textbf{Method Documentation:} \\
No Doc\newline
\textbf{Arguments:}
\begin{description}
\vspace{-.15cm}
\item[] \hspace{.5cm} \verb"event": No Doc
\end{description}
Example Call:
\begin{align*}
\mathbf{XDIConsole}-\hspace{-.15cm}>\mathbf{edit\_ports}, \ &event
\end{align*}
\begin{center}\rule{.85\textwidth}{.01cm}\end{center}
\subsubsection*{(pro) EDIT\_SETTINGS \label{XDIConsole::edit_settings_lab}}
\textbf{Method Documentation:} \\
No Doc\newline
\textbf{Arguments:}
\begin{description}
\vspace{-.15cm}
\item[] \hspace{.5cm} \verb"event": No Doc
\end{description}
Example Call:
\begin{align*}
\mathbf{XDIConsole}-\hspace{-.15cm}>\mathbf{edit\_settings}, \ &event
\end{align*}
\begin{center}\rule{.85\textwidth}{.01cm}\end{center}
\subsubsection*{(pro) EDITOR\_CLOSED \label{XDIConsole::editor_closed_lab}}
\textbf{Method Documentation:} \\
No Doc\newline
\textbf{Arguments:}
\begin{description}
\vspace{-.15cm}
\item[] \hspace{.5cm} \verb"event": No Doc
\end{description}
Example Call:
\begin{align*}
\mathbf{XDIConsole}-\hspace{-.15cm}>\mathbf{editor\_closed}, \ &event
\end{align*}
\begin{center}\rule{.85\textwidth}{.01cm}\end{center}
\subsubsection*{(pro) END\_AUTO\_OBJECT \label{XDIConsole::end_auto_object_lab}}
\textbf{Method Documentation:} \\
No Doc\newline
\textbf{Arguments:}
\begin{description}
\vspace{-.15cm}
\item[] \hspace{.5cm} \verb"id": No Doc
\vspace{-.15cm}
\item[] \hspace{.5cm} \verb"ref": No Doc
\vspace{-.15cm}
\item[] \hspace{.5cm} \verb"kill=kill": No Doc
\end{description}
Example Call:
\begin{align*}
\mathbf{XDIConsole}-\hspace{-.15cm}>\mathbf{end\_auto\_object}, \ &id,\\ \ &ref,\\ \ &kill=kill
\end{align*}
\begin{center}\rule{.85\textwidth}{.01cm}\end{center}
\subsubsection*{(pro) EVENT\_HANDLER \label{XDIConsole::Event_Handler_lab}}
\textbf{Method Documentation:} \\
No Doc\newline
\textbf{Arguments:}
\begin{description}
\vspace{-.15cm}
\item[] \hspace{.5cm} \verb"event": No Doc
\end{description}
Example Call:
\begin{align*}
\mathbf{XDIConsole}-\hspace{-.15cm}>\mathbf{Event\_Handler}, \ &event
\end{align*}
\begin{center}\rule{.85\textwidth}{.01cm}\end{center}
\subsubsection*{(pro) EXECUTE\_SCHEDULE \label{XDIConsole::execute_schedule_lab}}
\textbf{Method Documentation:} \\
No Doc\newline
Takes no arguments \\
Example Call:
\begin{align*}
\mathbf{XDIConsole}-\hspace{-.15cm}>\mathbf{execute\_schedule}
\end{align*}
\begin{center}\rule{.85\textwidth}{.01cm}\end{center}
\subsubsection*{(pro) FILE\_CHANGE\_SCHED \label{XDIConsole::file_change_sched_lab}}
\textbf{Method Documentation:} \\
No Doc\newline
\textbf{Arguments:}
\begin{description}
\vspace{-.15cm}
\item[] \hspace{.5cm} \verb"event": No Doc
\end{description}
Example Call:
\begin{align*}
\mathbf{XDIConsole}-\hspace{-.15cm}>\mathbf{file\_change\_sched}, \ &event
\end{align*}
\begin{center}\rule{.85\textwidth}{.01cm}\end{center}
\subsubsection*{(pro) FILE\_RE\_INITIALIZE \label{XDIConsole::file_re_initialize_lab}}
\textbf{Method Documentation:} \\
No Doc\newline
\textbf{Arguments:}
\begin{description}
\vspace{-.15cm}
\item[] \hspace{.5cm} \verb"event": No Doc
\end{description}
Example Call:
\begin{align*}
\mathbf{XDIConsole}-\hspace{-.15cm}>\mathbf{file\_re\_initialize}, \ &event
\end{align*}
\begin{center}\rule{.85\textwidth}{.01cm}\end{center}
\subsubsection*{(pro) FILE\_SHOW \label{XDIConsole::file_show_lab}}
\textbf{Method Documentation:} \\
No Doc\newline
\textbf{Arguments:}
\begin{description}
\vspace{-.15cm}
\item[] \hspace{.5cm} \verb"event": No Doc
\end{description}
Example Call:
\begin{align*}
\mathbf{XDIConsole}-\hspace{-.15cm}>\mathbf{file\_show}, \ &event
\end{align*}
\begin{center}\rule{.85\textwidth}{.01cm}\end{center}
\subsubsection*{(pro) FILE\_SHOW\_SCHED \label{XDIConsole::file_show_sched_lab}}
\textbf{Method Documentation:} \\
No Doc\newline
\textbf{Arguments:}
\begin{description}
\vspace{-.15cm}
\item[] \hspace{.5cm} \verb"event": No Doc
\end{description}
Example Call:
\begin{align*}
\mathbf{XDIConsole}-\hspace{-.15cm}>\mathbf{file\_show\_sched}, \ &event
\end{align*}
\begin{center}\rule{.85\textwidth}{.01cm}\end{center}
\subsubsection*{(function) FORCE\_IMAGE\_UPDATE \label{XDIConsole::force_image_update_lab}}
\textbf{Method Documentation:} \\
No Doc\newline
Takes no arguments \\
Example Call:
\begin{align*}
result = \mathbf{XDIConsole}-\hspace{-.15cm}>\mathbf{force\_image\_update}(&)
\end{align*}
\begin{center}\rule{.85\textwidth}{.01cm}\end{center}
\subsubsection*{(pro) GET\_CAMERA\_TEMP \label{XDIConsole::get_camera_temp_lab}}
\textbf{Method Documentation:} \\
No Doc\newline
\textbf{Arguments:}
\begin{description}
\vspace{-.15cm}
\item[] \hspace{.5cm} \verb"temp": No Doc
\vspace{-.15cm}
\item[] \hspace{.5cm} \verb"temp_state": No Doc
\vspace{-.15cm}
\item[] \hspace{.5cm} \verb"set_point": No Doc
\end{description}
Example Call:
\begin{align*}
\mathbf{XDIConsole}-\hspace{-.15cm}>\mathbf{get\_camera\_temp}, \ &temp,\\ \ &temp_state,\\ \ &set_point
\end{align*}
\begin{center}\rule{.85\textwidth}{.01cm}\end{center}
\subsubsection*{(function) GET\_DEFAULT\_PATH \label{XDIConsole::get_default_path_lab}}
\textbf{Method Documentation:} \\
No Doc\newline
Takes no arguments \\
Example Call:
\begin{align*}
result = \mathbf{XDIConsole}-\hspace{-.15cm}>\mathbf{get\_default\_path}(&)
\end{align*}
\begin{center}\rule{.85\textwidth}{.01cm}\end{center}
\subsubsection*{(function) GET\_DLL\_NAME \label{XDIConsole::get_dll_name_lab}}
\textbf{Method Documentation:} \\
No Doc\newline
Takes no arguments \\
Example Call:
\begin{align*}
result = \mathbf{XDIConsole}-\hspace{-.15cm}>\mathbf{get\_dll\_name}(&)
\end{align*}
\begin{center}\rule{.85\textwidth}{.01cm}\end{center}
\subsubsection*{(function) GET\_ETALON\_INFO \label{XDIConsole::get_etalon_info_lab}}
\textbf{Method Documentation:} \\
No Doc\newline
Takes no arguments \\
Example Call:
\begin{align*}
result = \mathbf{XDIConsole}-\hspace{-.15cm}>\mathbf{get\_etalon\_info}(&)
\end{align*}
\begin{center}\rule{.85\textwidth}{.01cm}\end{center}
\subsubsection*{(function) GET\_HEADER\_INFO \label{XDIConsole::get_header_info_lab}}
\textbf{Method Documentation:} \\
No Doc\newline
Takes no arguments \\
Example Call:
\begin{align*}
result = \mathbf{XDIConsole}-\hspace{-.15cm}>\mathbf{get\_header\_info}(&)
\end{align*}
\begin{center}\rule{.85\textwidth}{.01cm}\end{center}
\subsubsection*{(function) GET\_IMAGE \label{XDIConsole::get_image_lab}}
\textbf{Method Documentation:} \\
No Doc\newline
\textbf{Arguments:}
\begin{description}
\vspace{-.15cm}
\item[] \hspace{.5cm} \verb"image": No Doc
\end{description}
Example Call:
\begin{align*}
result = \mathbf{XDIConsole}-\hspace{-.15cm}>\mathbf{get\_image}(&image)
\end{align*}
\begin{center}\rule{.85\textwidth}{.01cm}\end{center}
\subsubsection*{(function) GET\_LOGGING\_INFO \label{XDIConsole::get_logging_info_lab}}
\textbf{Method Documentation:} \\
No Doc\newline
Takes no arguments \\
Example Call:
\begin{align*}
result = \mathbf{XDIConsole}-\hspace{-.15cm}>\mathbf{get\_logging\_info}(&)
\end{align*}
\begin{center}\rule{.85\textwidth}{.01cm}\end{center}
\subsubsection*{(function) GET\_PALETTE \label{XDIConsole::get_palette_lab}}
\textbf{Method Documentation:} \\
No Doc\newline
Takes no arguments \\
Example Call:
\begin{align*}
result = \mathbf{XDIConsole}-\hspace{-.15cm}>\mathbf{get\_palette}(&)
\end{align*}
\begin{center}\rule{.85\textwidth}{.01cm}\end{center}
\subsubsection*{(function) GET\_PHASE\_MAP\_PATH \label{XDIConsole::get_phase_map_path_lab}}
\textbf{Method Documentation:} \\
No Doc\newline
Takes no arguments \\
Example Call:
\begin{align*}
result = \mathbf{XDIConsole}-\hspace{-.15cm}>\mathbf{get\_phase\_map\_path}(&)
\end{align*}
\begin{center}\rule{.85\textwidth}{.01cm}\end{center}
\subsubsection*{(pro) GET\_PHASEMAP \label{XDIConsole::get_phasemap_lab}}
\textbf{Method Documentation:} \\
No Doc\newline
\textbf{Arguments:}
\begin{description}
\vspace{-.15cm}
\item[] \hspace{.5cm} \verb"phasemap_base": No Doc
\vspace{-.15cm}
\item[] \hspace{.5cm} \verb"phasemap_grad": No Doc
\vspace{-.15cm}
\item[] \hspace{.5cm} \verb"phasemap_lambda": No Doc
\end{description}
Example Call:
\begin{align*}
\mathbf{XDIConsole}-\hspace{-.15cm}>\mathbf{get\_phasemap}, \ &phasemap_base,\\ \ &phasemap_grad,\\ \ &phasemap_lambda
\end{align*}
\begin{center}\rule{.85\textwidth}{.01cm}\end{center}
\subsubsection*{(function) GET\_PORT\_MAP \label{XDIConsole::get_port_map_lab}}
\textbf{Method Documentation:} \\
No Doc\newline
Takes no arguments \\
Example Call:
\begin{align*}
result = \mathbf{XDIConsole}-\hspace{-.15cm}>\mathbf{get\_port\_map}(&)
\end{align*}
\begin{center}\rule{.85\textwidth}{.01cm}\end{center}
\subsubsection*{(function) GET\_RAW\_IMAGE \label{XDIConsole::get_raw_image_lab}}
\textbf{Method Documentation:} \\
No Doc\newline
\textbf{Arguments:}
\begin{description}
\vspace{-.15cm}
\item[] \hspace{.5cm} \verb"image": No Doc
\end{description}
Example Call:
\begin{align*}
result = \mathbf{XDIConsole}-\hspace{-.15cm}>\mathbf{get\_raw\_image}(&image)
\end{align*}
\begin{center}\rule{.85\textwidth}{.01cm}\end{center}
\subsubsection*{(function) GET\_SNR\_PER\_SCAN \label{XDIConsole::get_snr_per_scan_lab}}
\textbf{Method Documentation:} \\
No Doc\newline
Takes no arguments \\
Example Call:
\begin{align*}
result = \mathbf{XDIConsole}-\hspace{-.15cm}>\mathbf{get\_snr\_per\_scan}(&)
\end{align*}
\begin{center}\rule{.85\textwidth}{.01cm}\end{center}
\subsubsection*{(pro) GET\_SOURCE\_MAP \label{XDIConsole::get_source_map_lab}}
\textbf{Method Documentation:} \\
No Doc\newline
\textbf{Arguments:}
\begin{description}
\vspace{-.15cm}
\item[] \hspace{.5cm} \verb"smap": No Doc
\end{description}
Example Call:
\begin{align*}
\mathbf{XDIConsole}-\hspace{-.15cm}>\mathbf{get\_source\_map}, \ &smap
\end{align*}
\begin{center}\rule{.85\textwidth}{.01cm}\end{center}
\subsubsection*{(function) GET\_SPEC\_SAVE\_INFO \label{XDIConsole::get_spec_save_info_lab}}
\textbf{Method Documentation:} \\
No Doc\newline
\textbf{Arguments:}
\begin{description}
\vspace{-.15cm}
\item[] \hspace{.5cm} \verb"nrings": No Doc
\end{description}
Example Call:
\begin{align*}
result = \mathbf{XDIConsole}-\hspace{-.15cm}>\mathbf{get\_spec\_save\_info}(&nrings)
\end{align*}
\begin{center}\rule{.85\textwidth}{.01cm}\end{center}
\subsubsection*{(function) GET\_SPECTRA\_PATH \label{XDIConsole::get_spectra_path_lab}}
\textbf{Method Documentation:} \\
No Doc\newline
Takes no arguments \\
Example Call:
\begin{align*}
result = \mathbf{XDIConsole}-\hspace{-.15cm}>\mathbf{get\_spectra\_path}(&)
\end{align*}
\begin{center}\rule{.85\textwidth}{.01cm}\end{center}
\subsubsection*{(function) GET\_TIME\_NAME\_FORMAT \label{XDIConsole::get_time_name_format_lab}}
\textbf{Method Documentation:} \\
No Doc\newline
Takes no arguments \\
Example Call:
\begin{align*}
result = \mathbf{XDIConsole}-\hspace{-.15cm}>\mathbf{get\_time\_name\_format}(&)
\end{align*}
\begin{center}\rule{.85\textwidth}{.01cm}\end{center}
\subsubsection*{(function) GET\_ZONE\_SET\_PATH \label{XDIConsole::get_zone_set_path_lab}}
\textbf{Method Documentation:} \\
No Doc\newline
Takes no arguments \\
Example Call:
\begin{align*}
result = \mathbf{XDIConsole}-\hspace{-.15cm}>\mathbf{get\_zone\_set\_path}(&)
\end{align*}
\begin{center}\rule{.85\textwidth}{.01cm}\end{center}
\subsubsection*{(pro) IMAGE\_CAPTURE \label{XDIConsole::image_capture_lab}}
\textbf{Method Documentation:} \\
No Doc\newline
\textbf{Arguments:}
\begin{description}
\vspace{-.15cm}
\item[] \hspace{.5cm} \verb"event": No Doc
\end{description}
Example Call:
\begin{align*}
\mathbf{XDIConsole}-\hspace{-.15cm}>\mathbf{image\_capture}, \ &event
\end{align*}
\begin{center}\rule{.85\textwidth}{.01cm}\end{center}
\subsubsection*{(pro) KILL\_HANDLER \label{XDIConsole::Kill_Handler_lab}}
\textbf{Method Documentation:} \\
No Doc\newline
\textbf{Arguments:}
\begin{description}
\vspace{-.15cm}
\item[] \hspace{.5cm} \verb"id": No Doc
\vspace{-.15cm}
\item[] \hspace{.5cm} \verb"kill_widget=kill_widget": No Doc
\end{description}
Example Call:
\begin{align*}
\mathbf{XDIConsole}-\hspace{-.15cm}>\mathbf{Kill\_Handler}, \ &id,\\ \ &kill_widget=kill_widget
\end{align*}
\begin{center}\rule{.85\textwidth}{.01cm}\end{center}
\subsubsection*{(pro) LOAD\_SETTINGS \label{XDIConsole::load_settings_lab}}
\textbf{Method Documentation:} \\
No Doc\newline
\textbf{Arguments:}
\begin{description}
\vspace{-.15cm}
\item[] \hspace{.5cm} \verb"event": No Doc
\vspace{-.15cm}
\item[] \hspace{.5cm} \verb"filename=filename": No Doc
\vspace{-.15cm}
\item[] \hspace{.5cm} \verb"error=error": No Doc
\vspace{-.15cm}
\item[] \hspace{.5cm} \verb"first_call=first_call": No Doc
\end{description}
Example Call:
\begin{align*}
\mathbf{XDIConsole}-\hspace{-.15cm}>\mathbf{load\_settings}, \ &event,\\ \ &filename=filename,\\ \ &error=error,\\ \ &first_call=first_call
\end{align*}
\begin{center}\rule{.85\textwidth}{.01cm}\end{center}
\subsubsection*{(pro) LOG \label{XDIConsole::log_lab}}
\textbf{Method Documentation:} \\
No Doc\newline
\textbf{Arguments:}
\begin{description}
\vspace{-.15cm}
\item[] \hspace{.5cm} \verb"entry": No Doc
\vspace{-.15cm}
\item[] \hspace{.5cm} \verb"sender": No Doc
\vspace{-.15cm}
\item[] \hspace{.5cm} \verb"display_entry=display_entry": No Doc
\end{description}
Example Call:
\begin{align*}
\mathbf{XDIConsole}-\hspace{-.15cm}>\mathbf{log}, \ &entry,\\ \ &sender,\\ \ &display_entry=display_entry
\end{align*}
\begin{center}\rule{.85\textwidth}{.01cm}\end{center}
\subsubsection*{(pro) MODE\_SWITCH \label{XDIConsole::mode_switch_lab}}
\textbf{Method Documentation:} \\
No Doc\newline
\textbf{Arguments:}
\begin{description}
\vspace{-.15cm}
\item[] \hspace{.5cm} \verb"event": No Doc
\end{description}
Example Call:
\begin{align*}
\mathbf{XDIConsole}-\hspace{-.15cm}>\mathbf{mode\_switch}, \ &event
\end{align*}
\begin{center}\rule{.85\textwidth}{.01cm}\end{center}
\subsubsection*{(pro) MOT\_DRIVE\_CAL \label{XDIConsole::mot_drive_cal_lab}}
\textbf{Method Documentation:} \\
No Doc\newline
\textbf{Arguments:}
\begin{description}
\vspace{-.15cm}
\item[] \hspace{.5cm} \verb"event": No Doc
\end{description}
Example Call:
\begin{align*}
\mathbf{XDIConsole}-\hspace{-.15cm}>\mathbf{mot\_drive\_cal}, \ &event
\end{align*}
\begin{center}\rule{.85\textwidth}{.01cm}\end{center}
\subsubsection*{(pro) MOT\_DRIVE\_SKY \label{XDIConsole::mot_drive_sky_lab}}
\textbf{Method Documentation:} \\
No Doc\newline
\textbf{Arguments:}
\begin{description}
\vspace{-.15cm}
\item[] \hspace{.5cm} \verb"event": No Doc
\end{description}
Example Call:
\begin{align*}
\mathbf{XDIConsole}-\hspace{-.15cm}>\mathbf{mot\_drive\_sky}, \ &event
\end{align*}
\begin{center}\rule{.85\textwidth}{.01cm}\end{center}
\subsubsection*{(pro) MOT\_HOME\_CAL \label{XDIConsole::mot_home_cal_lab}}
\textbf{Method Documentation:} \\
No Doc\newline
\textbf{Arguments:}
\begin{description}
\vspace{-.15cm}
\item[] \hspace{.5cm} \verb"event": No Doc
\end{description}
Example Call:
\begin{align*}
\mathbf{XDIConsole}-\hspace{-.15cm}>\mathbf{mot\_home\_cal}, \ &event
\end{align*}
\begin{center}\rule{.85\textwidth}{.01cm}\end{center}
\subsubsection*{(pro) MOT\_HOME\_SKY \label{XDIConsole::mot_home_sky_lab}}
\textbf{Method Documentation:} \\
No Doc\newline
\textbf{Arguments:}
\begin{description}
\vspace{-.15cm}
\item[] \hspace{.5cm} \verb"event": No Doc
\end{description}
Example Call:
\begin{align*}
\mathbf{XDIConsole}-\hspace{-.15cm}>\mathbf{mot\_home\_sky}, \ &event
\end{align*}
\begin{center}\rule{.85\textwidth}{.01cm}\end{center}
\subsubsection*{(pro) MOT\_HOME\_SOURCE \label{XDIConsole::mot_home_source_lab}}
\textbf{Method Documentation:} \\
No Doc\newline
\textbf{Arguments:}
\begin{description}
\vspace{-.15cm}
\item[] \hspace{.5cm} \verb"image": No Doc
\end{description}
Example Call:
\begin{align*}
\mathbf{XDIConsole}-\hspace{-.15cm}>\mathbf{mot\_home\_source}, \ &image
\end{align*}
\begin{center}\rule{.85\textwidth}{.01cm}\end{center}
\subsubsection*{(pro) MOT\_SEL\_CAL \label{XDIConsole::mot_sel_cal_lab}}
\textbf{Method Documentation:} \\
No Doc\newline
\textbf{Arguments:}
\begin{description}
\vspace{-.15cm}
\item[] \hspace{.5cm} \verb"event": No Doc
\vspace{-.15cm}
\item[] \hspace{.5cm} \verb"set_source=set_source": No Doc
\end{description}
Example Call:
\begin{align*}
\mathbf{XDIConsole}-\hspace{-.15cm}>\mathbf{mot\_sel\_cal}, \ &event,\\ \ &set_source=set_source
\end{align*}
\begin{center}\rule{.85\textwidth}{.01cm}\end{center}
\subsubsection*{(pro) MOT\_SEL\_FILTER \label{XDIConsole::mot_sel_filter_lab}}
\textbf{Method Documentation:} \\
No Doc\newline
\textbf{Arguments:}
\begin{description}
\vspace{-.15cm}
\item[] \hspace{.5cm} \verb"event": No Doc
\end{description}
Example Call:
\begin{align*}
\mathbf{XDIConsole}-\hspace{-.15cm}>\mathbf{mot\_sel\_filter}, \ &event
\end{align*}
\begin{center}\rule{.85\textwidth}{.01cm}\end{center}
\subsubsection*{(pro) OPEN\_MPORT \label{XDIConsole::open_mport_lab}}
\textbf{Method Documentation:} \\
No Doc\newline
\textbf{Arguments:}
\begin{description}
\vspace{-.15cm}
\item[] \hspace{.5cm} \verb"event": No Doc
\end{description}
Example Call:
\begin{align*}
\mathbf{XDIConsole}-\hspace{-.15cm}>\mathbf{open\_mport}, \ &event
\end{align*}
\begin{center}\rule{.85\textwidth}{.01cm}\end{center}
\subsubsection*{(pro) REFRESH\_SPEC\_PMAPS \label{XDIConsole::refresh_spec_pmaps_lab}}
\textbf{Method Documentation:} \\
No Doc\newline
Takes no arguments \\
Example Call:
\begin{align*}
\mathbf{XDIConsole}-\hspace{-.15cm}>\mathbf{refresh\_spec\_pmaps}
\end{align*}
\begin{center}\rule{.85\textwidth}{.01cm}\end{center}
\subsubsection*{(pro) SAVE\_CURRENT\_SETTINGS \label{XDIConsole::save_current_settings_lab}}
\textbf{Method Documentation:} \\
No Doc\newline
\textbf{Arguments:}
\begin{description}
\vspace{-.15cm}
\item[] \hspace{.5cm} \verb"filename=filename": No Doc
\end{description}
Example Call:
\begin{align*}
\mathbf{XDIConsole}-\hspace{-.15cm}>\mathbf{save\_current\_settings}, \ &filename=filename
\end{align*}
\begin{center}\rule{.85\textwidth}{.01cm}\end{center}
\subsubsection*{(pro) SCAN\_ETALON \label{XDIConsole::scan_etalon_lab}}
\textbf{Method Documentation:} \\
No Doc\newline
\textbf{Arguments:}
\begin{description}
\vspace{-.15cm}
\item[] \hspace{.5cm} \verb"caller": No Doc
\vspace{-.15cm}
\item[] \hspace{.5cm} \verb"start_scan=start_scan": No Doc
\vspace{-.15cm}
\item[] \hspace{.5cm} \verb"stop_scan=stop_scan": No Doc
\vspace{-.15cm}
\item[] \hspace{.5cm} \verb"pause_scan=pause_scan": No Doc
\vspace{-.15cm}
\item[] \hspace{.5cm} \verb"cont_scan=cont_scan": No Doc
\vspace{-.15cm}
\item[] \hspace{.5cm} \verb"start_volt_offset=start_volt_offset": No Doc
\vspace{-.15cm}
\item[] \hspace{.5cm} \verb"stop_volt_offset=stop_volt_offset": No Doc
\vspace{-.15cm}
\item[] \hspace{.5cm} \verb"volt_step_size=volt_step_size": No Doc
\vspace{-.15cm}
\item[] \hspace{.5cm} \verb"status=status": No Doc
\vspace{-.15cm}
\item[] \hspace{.5cm} \verb"reference=reference": No Doc
\vspace{-.15cm}
\item[] \hspace{.5cm} \verb"get_ref=get_ref": No Doc
\vspace{-.15cm}
\item[] \hspace{.5cm} \verb"wavelength=wavelength": No Doc
\vspace{-.15cm}
\item[] \hspace{.5cm} \verb"force_start=force_start": No Doc
\end{description}
Example Call:
\begin{align*}
\mathbf{XDIConsole}-\hspace{-.15cm}>\mathbf{scan\_etalon}, \ &caller,\\ \ &start_scan=start_scan,\\ \ &stop_scan=stop_scan,\\ \ &pause_scan=pause_scan,\\ \ &cont_scan=cont_scan,\\ \ &start_volt_offset=start_volt_offset,\\ \ &stop_volt_offset=stop_volt_offset,\\ \ &volt_step_size=volt_step_size,\\ \ &status=status,\\ \ &reference=reference,\\ \ &get_ref=get_ref,\\ \ &wavelength=wavelength,\\ \ &force_start=force_start
\end{align*}
\begin{center}\rule{.85\textwidth}{.01cm}\end{center}
\subsubsection*{(pro) SEE\_CALIBRATION \label{XDIConsole::see_calibration_lab}}
\textbf{Method Documentation:} \\
No Doc\newline
\textbf{Arguments:}
\begin{description}
\vspace{-.15cm}
\item[] \hspace{.5cm} \verb"event": No Doc
\end{description}
Example Call:
\begin{align*}
\mathbf{XDIConsole}-\hspace{-.15cm}>\mathbf{see\_calibration}, \ &event
\end{align*}
\begin{center}\rule{.85\textwidth}{.01cm}\end{center}
\subsubsection*{(pro) SET\_CENTER \label{XDIConsole::set_center_lab}}
\textbf{Method Documentation:} \\
No Doc\newline
\textbf{Arguments:}
\begin{description}
\vspace{-.15cm}
\item[] \hspace{.5cm} \verb"xcen": No Doc
\vspace{-.15cm}
\item[] \hspace{.5cm} \verb"ycen": No Doc
\end{description}
Example Call:
\begin{align*}
\mathbf{XDIConsole}-\hspace{-.15cm}>\mathbf{set\_center}, \ &xcen,\\ \ &ycen
\end{align*}
\begin{center}\rule{.85\textwidth}{.01cm}\end{center}
\subsubsection*{(pro) SET\_NM\_PER\_STEP \label{XDIConsole::set_nm_per_step_lab}}
\textbf{Method Documentation:} \\
No Doc\newline
\textbf{Arguments:}
\begin{description}
\vspace{-.15cm}
\item[] \hspace{.5cm} \verb"nm_per_step": No Doc
\end{description}
Example Call:
\begin{align*}
\mathbf{XDIConsole}-\hspace{-.15cm}>\mathbf{set\_nm\_per\_step}, \ &nm_per_step
\end{align*}
\begin{center}\rule{.85\textwidth}{.01cm}\end{center}
\subsubsection*{(pro) SET\_PHASEMAP \label{XDIConsole::set_phasemap_lab}}
\textbf{Method Documentation:} \\
No Doc\newline
\textbf{Arguments:}
\begin{description}
\vspace{-.15cm}
\item[] \hspace{.5cm} \verb"phasemap_base": No Doc
\vspace{-.15cm}
\item[] \hspace{.5cm} \verb"phasemap_grad": No Doc
\vspace{-.15cm}
\item[] \hspace{.5cm} \verb"phasemap_lambda": No Doc
\end{description}
Example Call:
\begin{align*}
\mathbf{XDIConsole}-\hspace{-.15cm}>\mathbf{set\_phasemap}, \ &phasemap_base,\\ \ &phasemap_grad,\\ \ &phasemap_lambda
\end{align*}
\begin{center}\rule{.85\textwidth}{.01cm}\end{center}
\subsubsection*{(pro) SET\_SNR\_PER\_SCAN \label{XDIConsole::set_snr_per_scan_lab}}
\textbf{Method Documentation:} \\
No Doc\newline
\textbf{Arguments:}
\begin{description}
\vspace{-.15cm}
\item[] \hspace{.5cm} \verb"snr": No Doc
\end{description}
Example Call:
\begin{align*}
\mathbf{XDIConsole}-\hspace{-.15cm}>\mathbf{set\_snr\_per\_scan}, \ &snr
\end{align*}
\begin{center}\rule{.85\textwidth}{.01cm}\end{center}
\subsubsection*{(pro) SET\_SOURCE\_MAP \label{XDIConsole::set_source_map_lab}}
\textbf{Method Documentation:} \\
No Doc\newline
\textbf{Arguments:}
\begin{description}
\vspace{-.15cm}
\item[] \hspace{.5cm} \verb"smap": No Doc
\end{description}
Example Call:
\begin{align*}
\mathbf{XDIConsole}-\hspace{-.15cm}>\mathbf{set\_source\_map}, \ &smap
\end{align*}
\begin{center}\rule{.85\textwidth}{.01cm}\end{center}
\subsubsection*{(pro) SHUTDOWN\_SPEX \label{XDIConsole::shutdown_spex_lab}}
\textbf{Method Documentation:} \\
No Doc\newline
Takes no arguments \\
Example Call:
\begin{align*}
\mathbf{XDIConsole}-\hspace{-.15cm}>\mathbf{shutdown\_spex}
\end{align*}
\begin{center}\rule{.85\textwidth}{.01cm}\end{center}
\subsubsection*{(pro) SPECTRUM\_SNAPSHOT \label{XDIConsole::spectrum_snapshot_lab}}
\textbf{Method Documentation:} \\
No Doc\newline
\textbf{Arguments:}
\begin{description}
\vspace{-.15cm}
\item[] \hspace{.5cm} \verb"snapshot": No Doc
\end{description}
Example Call:
\begin{align*}
\mathbf{XDIConsole}-\hspace{-.15cm}>\mathbf{spectrum\_snapshot}, \ &snapshot
\end{align*}
\begin{center}\rule{.85\textwidth}{.01cm}\end{center}
\subsubsection*{(pro) START\_PLUGIN \label{XDIConsole::start_plugin_lab}}
\textbf{Method Documentation:} \\
No Doc\newline
\textbf{Arguments:}
\begin{description}
\vspace{-.15cm}
\item[] \hspace{.5cm} \verb"event": No Doc
\vspace{-.15cm}
\item[] \hspace{.5cm} \verb"args=args": No Doc
\vspace{-.15cm}
\item[] \hspace{.5cm} \verb"new_obj=new_obj": No Doc
\end{description}
Example Call:
\begin{align*}
\mathbf{XDIConsole}-\hspace{-.15cm}>\mathbf{start\_plugin}, \ &event,\\ \ &args=args,\\ \ &new_obj=new_obj
\end{align*}
\begin{center}\rule{.85\textwidth}{.01cm}\end{center}
\subsubsection*{(pro) TIMER\_EVENT \label{XDIConsole::timer_event_lab}}
\textbf{Method Documentation:} \\
No Doc\newline
Takes no arguments \\
Example Call:
\begin{align*}
\mathbf{XDIConsole}-\hspace{-.15cm}>\mathbf{timer\_event}
\end{align*}
\begin{center}\rule{.85\textwidth}{.01cm}\end{center}
\subsubsection*{(pro) UPDATE\_CAMERA \label{XDIConsole::update_camera_lab}}
\textbf{Method Documentation:} \\
No Doc\newline
\textbf{Arguments:}
\begin{description}
\vspace{-.15cm}
\item[] \hspace{.5cm} \verb"commands": No Doc
\vspace{-.15cm}
\item[] \hspace{.5cm} \verb"results": No Doc
\end{description}
Example Call:
\begin{align*}
\mathbf{XDIConsole}-\hspace{-.15cm}>\mathbf{update\_camera}, \ &commands,\\ \ &results
\end{align*}
\begin{center}\rule{.85\textwidth}{.01cm}\end{center}
\subsubsection*{(pro) UPDATE\_LEGS \label{XDIConsole::update_legs_lab}}
\textbf{Method Documentation:} \\
No Doc\newline
\textbf{Arguments:}
\begin{description}
\vspace{-.15cm}
\item[] \hspace{.5cm} \verb"leg1=leg1": No Doc
\vspace{-.15cm}
\item[] \hspace{.5cm} \verb"leg2=leg2": No Doc
\vspace{-.15cm}
\item[] \hspace{.5cm} \verb"leg3=leg3": No Doc
\vspace{-.15cm}
\item[] \hspace{.5cm} \verb"legs=legs": No Doc
\end{description}
Example Call:
\begin{align*}
\mathbf{XDIConsole}-\hspace{-.15cm}>\mathbf{update\_legs}, \ &leg1=leg1,\\ \ &leg2=leg2,\\ \ &leg3=leg3,\\ \ &legs=legs
\end{align*}
\begin{center}\rule{.85\textwidth}{.01cm}\end{center}
\newpage
\section*{XDILog \label{XDILog__define_lab}}
\addcontentsline{toc}{section}{XDILog}
The Log class manages writing log output, both to the console log window and to a text file.\newline \newline
Inherits from: \textbf{None} \newline
Class Data: 
\begin{table}[!h]
\begin{tiny}\vspace{-.1cm}\begin{center}
\begin{tabular}{rl|rl|rl}
\hline
(\verb"string") & log& (\verb"long") & log\_window& (\verb"string") & prog\_name \\
(\verb"string") & log\_path& (\verb"int") & show\_log& (\verb"string") & curdate \\
\hline
\end{tabular}\end{center}\end{tiny}\end{table}\vspace{-.5cm} \\
\textbf{Defined in file:} \newline
\small{C:/cal/Operations/SDI\_Instruments/common/idl/core/XDILog\_\_define.pro}
\begin{center}\rule{1\textwidth}{.02cm}\end{center}
\subsection*{METHODS:}
\subsubsection*{(function) INIT \label{XDILog::init_lab}}
\textbf{Method Documentation:} \\
Initialize the log.\newline
\textbf{Arguments:}
\begin{description}
\vspace{-.15cm}
\item[] \hspace{.5cm} \verb"log_window=log_window": Widget window for the log, optional
\vspace{-.15cm}
\item[] \hspace{.5cm} \verb"show_log=show_log": Show the log
\vspace{-.15cm}
\item[] \hspace{.5cm} \verb"prog_name=prog_name": The name of the plugin, used for naming log files
\vspace{-.15cm}
\item[] \hspace{.5cm} \verb"log_path=log_path": Path to store the log files
\vspace{-.15cm}
\item[] \hspace{.5cm} \verb"log_append=log_append": Append to existing logs
\vspace{-.15cm}
\item[] \hspace{.5cm} \verb"enabled=enabled": Is logging enabled?
\vspace{-.15cm}
\item[] \hspace{.5cm} \verb"header=header": A header for the log file, used when creating a new log
\end{description}
Example Call:
\begin{align*}
result = \mathbf{XDILog}-\hspace{-.15cm}>\mathbf{init}(&log_window=log_window,\\ \ &show_log=show_log,\\ \ &prog_name=prog_name,\\ \ &log_path=log_path,\\ \ &log_append=log_append,\\ \ &enabled=enabled,\\ \ &header=header)
\end{align*}
\begin{center}\rule{.85\textwidth}{.01cm}\end{center}
\subsubsection*{(pro) REFRESH \label{XDILog::refresh_lab}}
\textbf{Method Documentation:} \\
Refresh the log window with the current log contents.\newline
Takes no arguments \\
Example Call:
\begin{align*}
\mathbf{XDILog}-\hspace{-.15cm}>\mathbf{refresh}
\end{align*}
\begin{center}\rule{.85\textwidth}{.01cm}\end{center}
\subsubsection*{(pro) UPDATE \label{XDILog::update_lab}}
\textbf{Method Documentation:} \\
Add an entry to the log, prepending a date/time string.\newline
\textbf{Arguments:}
\begin{description}
\vspace{-.15cm}
\item[] \hspace{.5cm} \verb"entry": String entry to add to the log
\end{description}
Example Call:
\begin{align*}
\mathbf{XDILog}-\hspace{-.15cm}>\mathbf{update}, \ &entry
\end{align*}
\begin{center}\rule{.85\textwidth}{.01cm}\end{center}
\newpage
\section*{XDIWidgetReg \label{XDIWidgetReg__define_lab}}
\addcontentsline{toc}{section}{XDIWidgetReg}
This class manages plugins, by storing their object references and names in a linked list and managing that list.\newline \newline
Inherits from: \textbf{None} \newline
Class Data: 
\begin{table}[!h]
\begin{tiny}\vspace{-.1cm}\begin{center}
\begin{tabular}{rl|rl|rl}
\hline
(\verb"long") & id& (\verb"string") & type& (\verb"obj") & ref \\
(\verb"int") & store& (\verb"int") & need\_timer& (\verb"int") & need\_frame \\
\hline
\end{tabular}\end{center}\end{tiny}\end{table}\vspace{-.5cm} \\
\textbf{Defined in file:} \newline
\small{C:/cal/Operations/SDI\_Instruments/common/idl/core/xdiwidgetreg\_\_define.pro}
\begin{center}\rule{1\textwidth}{.02cm}\end{center}
\subsection*{METHODS:}
\subsubsection*{(function) INIT \label{XDIWidgetReg::init_lab}}
\textbf{Method Documentation:} \\
Initialize the plugin list with the id and object reference of the console.\newline
\textbf{Arguments:}
\begin{description}
\vspace{-.15cm}
\item[] \hspace{.5cm} \verb"ref=ref": Console object reference
\vspace{-.15cm}
\item[] \hspace{.5cm} \verb"id=id": Console widget id
\end{description}
Example Call:
\begin{align*}
result = \mathbf{XDIWidgetReg}-\hspace{-.15cm}>\mathbf{init}(&ref=ref,\\ \ &id=id)
\end{align*}
\begin{center}\rule{.85\textwidth}{.01cm}\end{center}
\subsubsection*{(function) COUNT\_OBJECTS \label{XDIWidgetReg::count_objects_lab}}
\textbf{Method Documentation:} \\
Count the number of plugins, and return the count.\newline
Takes no arguments \\
Example Call:
\begin{align*}
result = \mathbf{XDIWidgetReg}-\hspace{-.15cm}>\mathbf{count\_objects}(&)
\end{align*}
\begin{center}\rule{.85\textwidth}{.01cm}\end{center}
\subsubsection*{(pro) DELETE\_INSTANCE \label{XDIWidgetReg::delete_instance_lab}}
\textbf{Method Documentation:} \\
Remove a plugin from the list.\newline
\textbf{Arguments:}
\begin{description}
\vspace{-.15cm}
\item[] \hspace{.5cm} \verb"id": Widget id of the plugin to remove
\end{description}
Example Call:
\begin{align*}
\mathbf{XDIWidgetReg}-\hspace{-.15cm}>\mathbf{delete\_instance}, \ &id
\end{align*}
\begin{center}\rule{.85\textwidth}{.01cm}\end{center}
\subsubsection*{(function) GENERATE\_LIST \label{XDIWidgetReg::generate_list_lab}}
\textbf{Method Documentation:} \\
Generate a structure whose fields are arrays, one element for each plugin, containing the plugin info.\newline
Takes no arguments \\
Example Call:
\begin{align*}
result = \mathbf{XDIWidgetReg}-\hspace{-.15cm}>\mathbf{generate\_list}(&)
\end{align*}
\begin{center}\rule{.85\textwidth}{.01cm}\end{center}
\subsubsection*{(function) MATCH\_REGISTER\_FRAME \label{XDIWidgetReg::match_register_frame_lab}}
\textbf{Method Documentation:} \\
From a widget id, return the need\_frame field of the plugin.\newline
\textbf{Arguments:}
\begin{description}
\vspace{-.15cm}
\item[] \hspace{.5cm} \verb"id": Widget if of the plugin's main window
\end{description}
Example Call:
\begin{align*}
result = \mathbf{XDIWidgetReg}-\hspace{-.15cm}>\mathbf{match\_register\_frame}(&id)
\end{align*}
\begin{center}\rule{.85\textwidth}{.01cm}\end{center}
\subsubsection*{(function) MATCH\_REGISTER\_FROM\_TYPE \label{XDIWidgetReg::match_register_from_type_lab}}
\textbf{Method Documentation:} \\
From a plugin type, return the object reference of the plugin.\newline
\textbf{Arguments:}
\begin{description}
\vspace{-.15cm}
\item[] \hspace{.5cm} \verb"type": String type of the plugin
\end{description}
Example Call:
\begin{align*}
result = \mathbf{XDIWidgetReg}-\hspace{-.15cm}>\mathbf{match\_register\_from\_type}(&type)
\end{align*}
\begin{center}\rule{.85\textwidth}{.01cm}\end{center}
\subsubsection*{(function) MATCH\_REGISTER\_REF \label{XDIWidgetReg::match_register_ref_lab}}
\textbf{Method Documentation:} \\
From a widget id, return the object reference of the plugin.\newline
\textbf{Arguments:}
\begin{description}
\vspace{-.15cm}
\item[] \hspace{.5cm} \verb"id": Widget if of the plugin's main window
\end{description}
Example Call:
\begin{align*}
result = \mathbf{XDIWidgetReg}-\hspace{-.15cm}>\mathbf{match\_register\_ref}(&id)
\end{align*}
\begin{center}\rule{.85\textwidth}{.01cm}\end{center}
\subsubsection*{(function) MATCH\_REGISTER\_STORE \label{XDIWidgetReg::match_register_store_lab}}
\textbf{Method Documentation:} \\
Given a widget id, return the value of the store field for the corresponding plugin.\newline
\textbf{Arguments:}
\begin{description}
\vspace{-.15cm}
\item[] \hspace{.5cm} \verb"id": Widget id of the plugin's main window
\end{description}
Example Call:
\begin{align*}
result = \mathbf{XDIWidgetReg}-\hspace{-.15cm}>\mathbf{match\_register\_store}(&id)
\end{align*}
\begin{center}\rule{.85\textwidth}{.01cm}\end{center}
\subsubsection*{(function) MATCH\_REGISTER\_TIMER \label{XDIWidgetReg::match_register_timer_lab}}
\textbf{Method Documentation:} \\
From a widget id, return the need\_frame field of the plugin.\newline
\textbf{Arguments:}
\begin{description}
\vspace{-.15cm}
\item[] \hspace{.5cm} \verb"id": Widget if of the plugin's main window
\end{description}
Example Call:
\begin{align*}
result = \mathbf{XDIWidgetReg}-\hspace{-.15cm}>\mathbf{match\_register\_timer}(&id)
\end{align*}
\begin{center}\rule{.85\textwidth}{.01cm}\end{center}
\subsubsection*{(function) MATCH\_REGISTER\_TYPE \label{XDIWidgetReg::match_register_type_lab}}
\textbf{Method Documentation:} \\
From a widget id, return the plugin type.\newline
\textbf{Arguments:}
\begin{description}
\vspace{-.15cm}
\item[] \hspace{.5cm} \verb"id": Widget id of the plugin's main window
\end{description}
Example Call:
\begin{align*}
result = \mathbf{XDIWidgetReg}-\hspace{-.15cm}>\mathbf{match\_register\_type}(&id)
\end{align*}
\begin{center}\rule{.85\textwidth}{.01cm}\end{center}
\subsubsection*{(pro) PRINT\_REGISTER \label{XDIWidgetReg::print_register_lab}}
\textbf{Method Documentation:} \\
Print out info about the list of plugins.\newline
Takes no arguments \\
Example Call:
\begin{align*}
\mathbf{XDIWidgetReg}-\hspace{-.15cm}>\mathbf{print\_register}
\end{align*}
\begin{center}\rule{.85\textwidth}{.01cm}\end{center}
\subsubsection*{(pro) REGISTER \label{XDIWidgetReg::register_lab}}
\textbf{Method Documentation:} \\
Add a plugin to the list.\newline
\textbf{Arguments:}
\begin{description}
\vspace{-.15cm}
\item[] \hspace{.5cm} \verb"id": Widget id of the plugin's main window
\vspace{-.15cm}
\item[] \hspace{.5cm} \verb"ref": Object reference for this instance of the plugin
\vspace{-.15cm}
\item[] \hspace{.5cm} \verb"type": Plugin type (string name)
\vspace{-.15cm}
\item[] \hspace{.5cm} \verb"store": Flag to indicate whether or not to save plugin settings
\vspace{-.15cm}
\item[] \hspace{.5cm} \verb"timer": Flag to indicate this plugin needs to recieve timer events
\vspace{-.15cm}
\item[] \hspace{.5cm} \verb"frame": Flag to indicate this plugin needs to recieve frame events
\end{description}
Example Call:
\begin{align*}
\mathbf{XDIWidgetReg}-\hspace{-.15cm}>\mathbf{register}, \ &id,\\ \ &ref,\\ \ &type,\\ \ &store,\\ \ &timer,\\ \ &frame
\end{align*}
\begin{center}\rule{.85\textwidth}{.01cm}\end{center}
\subsubsection*{(pro) SAVE\_SETTINGS \label{XDIWidgetReg::save_settings_lab}}
\textbf{Method Documentation:} \\
This implements the ability of plugins to save their settings, to be restored next time they are opened.\newline
\textbf{Arguments:}
\begin{description}
\vspace{-.15cm}
\item[] \hspace{.5cm} \verb"path": The settings save path
\vspace{-.15cm}
\item[] \hspace{.5cm} \verb"id": The widget id of the plugin's main window
\vspace{-.15cm}
\item[] \hspace{.5cm} \verb"owner": String name of the plugin
\vspace{-.15cm}
\item[] \hspace{.5cm} \verb"ref": Object reference to the plugin
\end{description}
Example Call:
\begin{align*}
\mathbf{XDIWidgetReg}-\hspace{-.15cm}>\mathbf{save\_settings}, \ &path,\\ \ &id,\\ \ &owner,\\ \ &ref
\end{align*}
\begin{center}\rule{.85\textwidth}{.01cm}\end{center}
\subsubsection*{(pro) SET\_CONTROL \label{XDIWidgetReg::set_control_lab}}
\textbf{Method Documentation:} \\
I have no idea what this does, and it does not appear to be called anywhere in the SDI code base, so it is probably a holdover from an early version.\newline
\textbf{Arguments:}
\begin{description}
\vspace{-.15cm}
\item[] \hspace{.5cm} \verb"id": No Doc
\vspace{-.15cm}
\item[] \hspace{.5cm} \verb"ref": No Doc
\vspace{-.15cm}
\item[] \hspace{.5cm} \verb"control": No Doc
\end{description}
Example Call:
\begin{align*}
\mathbf{XDIWidgetReg}-\hspace{-.15cm}>\mathbf{set\_control}, \ &id,\\ \ &ref,\\ \ &control
\end{align*}
\begin{center}\rule{.85\textwidth}{.01cm}\end{center}
\chapter*{Functions}
\addcontentsline{toc}{chapter}{Functions}
\section*{(function) GET\_ERROR \label{Get_Error_lab}}
\addcontentsline{toc}{section}{Get\_Error}
\textbf{Defined in file:} \newline
\small{C:/cal/Operations/SDI\_Instruments/common/idl/core/get\_error.pro} \newline
\textbf{Function Documentation:} \\
Return an ANDOR error string given an error code.\newline
\textbf{Arguments:}
\begin{description}
\vspace{-.15cm}
\item[] \hspace{.5cm} \verb"err_code": Error code
\end{description}
\begin{center}\rule{.85\textwidth}{.01cm}\end{center}
\section*{(function) GET\_NAMES \label{Get_Names_lab}}
\addcontentsline{toc}{section}{Get\_Names}
\textbf{Defined in file:} \newline
\small{C:/cal/Operations/SDI\_Instruments/common/idl/core/get\_names.pro} \newline
\textbf{Function Documentation:} \\
From a full path list of plugins, return only the plugin names\newline
\textbf{Arguments:}
\begin{description}
\vspace{-.15cm}
\item[] \hspace{.5cm} \verb"path_list": Vector of plugin full path names
\end{description}
\begin{center}\rule{.85\textwidth}{.01cm}\end{center}
\section*{(function) ACE\_FILTER\_INTERFACE \label{ace_filter_interface_lab}}
\addcontentsline{toc}{section}{ace\_filter\_interface}
\textbf{Defined in file:} \newline
\small{C:/cal/Operations/SDI\_Instruments/common/idl/core/ace\_filter\_interface.pro} \newline
\textbf{Function Documentation:} \\
Sends commands to an ACE filter wheel (used only at Poker I guess, since com ports are hard coded here.\newline
\textbf{Arguments:}
\begin{description}
\vspace{-.15cm}
\item[] \hspace{.5cm} \verb"command=command": Command to send
\end{description}
\begin{center}\rule{.85\textwidth}{.01cm}\end{center}
\section*{(function) DRIVE\_MOTOR \label{drive_motor_lab}}
\addcontentsline{toc}{section}{drive\_motor}
\textbf{Defined in file:} \newline
\small{C:/cal/Operations/SDI\_Instruments/common/idl/core/drive\_motor.pro} \newline
\textbf{Function Documentation:} \\
Wrapper for controlling Fualhaber motors. Open/close ports, enable/disable motor, get status, set position, drive to position, set speed/accel, drive in a direction in small increments until blocked (i.e. when homing the mirror motor) etc.\newline
\textbf{Arguments:}
\begin{description}
\vspace{-.15cm}
\item[] \hspace{.5cm} \verb"port": Com port of the motor
\vspace{-.15cm}
\item[] \hspace{.5cm} \verb"dll_name": SDI\_External dll name (full path)
\vspace{-.15cm}
\item[] \hspace{.5cm} \verb"direction=direction": String direction ("forwards" or "backwards") to drive until blocked
\vspace{-.15cm}
\item[] \hspace{.5cm} \verb"gohix=gohix": Drive to nearest hall index
\vspace{-.15cm}
\item[] \hspace{.5cm} \verb"goix=goix": 
\vspace{-.15cm}
\item[] \hspace{.5cm} \verb"drive_to=drive_to": Drive to absolute position
\vspace{-.15cm}
\item[] \hspace{.5cm} \verb"control=control": String control command (see function body)
\vspace{-.15cm}
\item[] \hspace{.5cm} \verb"readpos=readpos": Read the motor position (returned from the function)
\vspace{-.15cm}
\item[] \hspace{.5cm} \verb"speed=speed": Set the speed
\vspace{-.15cm}
\item[] \hspace{.5cm} \verb"accel=accel": Set the acceleration
\vspace{-.15cm}
\item[] \hspace{.5cm} \verb"verbatim=verbatim": Send a string command verbatim to the motor, appending a carriage return
\vspace{-.15cm}
\item[] \hspace{.5cm} \verb"home_max_spin_time=home_max_spin_time": Max time to spin (for every small increment) when homing
\vspace{-.15cm}
\item[] \hspace{.5cm} \verb"timeout=timeout": Timeout in seconds
\end{description}
\begin{center}\rule{.85\textwidth}{.01cm}\end{center}
\section*{(function) GET\_PATHS \label{get_paths_lab}}
\addcontentsline{toc}{section}{get\_paths}
\textbf{Defined in file:} \newline
\small{C:/cal/Operations/SDI\_Instruments/common/idl/core/get\_paths.pro} \newline
\textbf{Function Documentation:} \\
No Doc\newline
Takes no arguments \\
\begin{center}\rule{.85\textwidth}{.01cm}\end{center}
\section*{(function) GET\_SUN\_ELEVATION \label{get_sun_elevation_lab}}
\addcontentsline{toc}{section}{get\_sun\_elevation}
\textbf{Defined in file:} \newline
\small{C:/cal/Operations/SDI\_Instruments/common/idl/core/get\_sun\_elevation.pro} \newline
\textbf{Function Documentation:} \\
Get the current sun elevation for a given latitude and longitude.\newline
\textbf{Arguments:}
\begin{description}
\vspace{-.15cm}
\item[] \hspace{.5cm} \verb"lat": Geographic latitude
\vspace{-.15cm}
\item[] \hspace{.5cm} \verb"lon": Geographic longitude
\end{description}
\begin{center}\rule{.85\textwidth}{.01cm}\end{center}
\section*{(function) PHASEMAP\_UNWRAP \label{phasemap_unwrap_lab}}
\addcontentsline{toc}{section}{phasemap\_unwrap}
\textbf{Defined in file:} \newline
\small{C:/cal/Operations/SDI\_Instruments/common/idl/core/phasemap\_unwrap.pro} \newline
\textbf{Function Documentation:} \\
`Unwrap' a phasemap produced by the SDIPhasemapper plugin.\newline
\textbf{Arguments:}
\begin{description}
\vspace{-.15cm}
\item[] \hspace{.5cm} \verb"xcen": Nominal x center
\vspace{-.15cm}
\item[] \hspace{.5cm} \verb"ycen": Nominal y center
\vspace{-.15cm}
\item[] \hspace{.5cm} \verb"radial_chunk": Size of the chunk over which to average the phase (value of 50 is used in phasemapper)
\vspace{-.15cm}
\item[] \hspace{.5cm} \verb"channels": Number of channels in the scan
\vspace{-.15cm}
\item[] \hspace{.5cm} \verb"threshold": Value of 80 is used by the phasemapper
\vspace{-.15cm}
\item[] \hspace{.5cm} \verb"wavelength": The wavelength at which the phasemap was recorded
\vspace{-.15cm}
\item[] \hspace{.5cm} \verb"phasemap": The actual phasemap 2D array
\vspace{-.15cm}
\item[] \hspace{.5cm} \verb"show=show": Show the unwrap as it occurs
\vspace{-.15cm}
\item[] \hspace{.5cm} \verb"tv_id=tv_id": Id of the tv window for showing the unwrap
\vspace{-.15cm}
\item[] \hspace{.5cm} \verb"dims=dims": Dimensions of the tv window for drawing
\end{description}
\begin{center}\rule{.85\textwidth}{.01cm}\end{center}
\section*{(function) ZONEMAPPER \label{zonemapper_lab}}
\addcontentsline{toc}{section}{zonemapper}
\textbf{Defined in file:} \newline
\small{C:/cal/Operations/SDI\_Instruments/common/idl/core/zonemapper.pro} \newline
\textbf{Function Documentation:} \\
Creates a zone map, a 2D array of numbers indicating the zone number of each pixel.\newline
\textbf{Arguments:}
\begin{description}
\vspace{-.15cm}
\item[] \hspace{.5cm} \verb"nx": X dimension
\vspace{-.15cm}
\item[] \hspace{.5cm} \verb"ny": Y dimension
\vspace{-.15cm}
\item[] \hspace{.5cm} \verb"cent": 2-element vector containing x and y center pixels
\vspace{-.15cm}
\item[] \hspace{.5cm} \verb"rads": Vector containing the radius of each ring
\vspace{-.15cm}
\item[] \hspace{.5cm} \verb"secs": Vector containing the number of sectors in each ring
\vspace{-.15cm}
\item[] \hspace{.5cm} \verb"nums": This should be set to 0, it is not needed
\vspace{-.15cm}
\item[] \hspace{.5cm} \verb"show=show": Show the resulting zonemap
\vspace{-.15cm}
\item[] \hspace{.5cm} \verb"outang=outang": OUT: return the `azimuth' of each zone
\vspace{-.15cm}
\item[] \hspace{.5cm} \verb"outrad=outrad": OUT: return the radius of each zone
\end{description}
\begin{center}\rule{.85\textwidth}{.01cm}\end{center}
\chapter*{Procedures}
\addcontentsline{toc}{chapter}{Procedures}
\section*{(pro) GET\_EPHEMERIS \label{Get_Ephemeris_lab}}
\addcontentsline{toc}{section}{Get\_Ephemeris}
\textbf{Defined in file:} \newline
\small{C:/cal/Operations/SDI\_Instruments/common/idl/core/get\_ephemeris.pro} \newline
\textbf{Procedure Documentation:} \\
No Doc\newline
\textbf{Arguments:}
\begin{description}
\vspace{-.15cm}
\item[] \hspace{.5cm} \verb"save_name=save_name": No Doc
\vspace{-.15cm}
\item[] \hspace{.5cm} \verb"safe_sea=safe_sea": No Doc
\vspace{-.15cm}
\item[] \hspace{.5cm} \verb"lat=lat": No Doc
\vspace{-.15cm}
\item[] \hspace{.5cm} \verb"lon=lon": No Doc
\vspace{-.15cm}
\item[] \hspace{.5cm} \verb"timeres=timeres": No Doc
\vspace{-.15cm}
\item[] \hspace{.5cm} \verb"start_stop_times=start_stop_times": No Doc
\vspace{-.15cm}
\item[] \hspace{.5cm} \verb"get_sea=get_sea": No Doc
\end{description}
\begin{center}\rule{.85\textwidth}{.01cm}\end{center}
\section*{(pro) HANDLE\_ERROR \label{Handle_Error_lab}}
\addcontentsline{toc}{section}{Handle\_Error}
\textbf{Defined in file:} \newline
\small{C:/cal/Operations/SDI\_Instruments/common/idl/core/SDI\_Main.pro} \newline
\textbf{Procedure Documentation:} \\
Error handler.\newline
\textbf{Arguments:}
\begin{description}
\vspace{-.15cm}
\item[] \hspace{.5cm} \verb"error": Error recieved
\end{description}
\begin{center}\rule{.85\textwidth}{.01cm}\end{center}
\section*{(pro) HANDLE\_EVENT \label{Handle_Event_lab}}
\addcontentsline{toc}{section}{Handle\_Event}
\textbf{Defined in file:} \newline
\small{C:/cal/Operations/SDI\_Instruments/common/idl/core/SDI\_Main.pro} \newline
\textbf{Procedure Documentation:} \\
Handle widget events. These are rerouted to the console's event handler.\newline
\textbf{Arguments:}
\begin{description}
\vspace{-.15cm}
\item[] \hspace{.5cm} \verb"event": Widget event structure
\end{description}
\begin{center}\rule{.85\textwidth}{.01cm}\end{center}
\section*{(pro) KILL\_ENTRY \label{Kill_Entry_lab}}
\addcontentsline{toc}{section}{Kill\_Entry}
\textbf{Defined in file:} \newline
\small{C:/cal/Operations/SDI\_Instruments/common/idl/core/SDI\_Main.pro} \newline
\textbf{Procedure Documentation:} \\
Handle widget destroy events. These are rerouted to the consoles kill handler.\newline
\textbf{Arguments:}
\begin{description}
\vspace{-.15cm}
\item[] \hspace{.5cm} \verb"id": Widget id
\end{description}
\begin{center}\rule{.85\textwidth}{.01cm}\end{center}
\section*{(pro) MARKS\_PALETTE \label{MARKS_PALETTE_lab}}
\addcontentsline{toc}{section}{MARKS\_PALETTE}
\textbf{Defined in file:} \newline
\small{C:/cal/Operations/SDI\_Instruments/common/idl/core/load\_pal.pro} \newline
\textbf{Procedure Documentation:} \\
No Doc\newline
Takes no arguments \\
\begin{center}\rule{.85\textwidth}{.01cm}\end{center}
\section*{(pro) SDI\_MAIN \label{SDI_Main_lab}}
\addcontentsline{toc}{section}{SDI\_Main}
\textbf{Defined in file:} \newline
\small{C:/cal/Operations/SDI\_Instruments/common/idl/core/SDI\_Main.pro} \newline
\textbf{Procedure Documentation:} \\
SDI entry point, called with a settings file, optional schedule and optional mode.\newline
\textbf{Arguments:}
\begin{description}
\vspace{-.15cm}
\item[] \hspace{.5cm} \verb"settings=settings": Settings file (required)
\vspace{-.15cm}
\item[] \hspace{.5cm} \verb"schedule=schedule": Schedule file (required if mode is "auto")
\vspace{-.15cm}
\item[] \hspace{.5cm} \verb"mode=mode": String mode, "auto" or "manual", defaults to "manual"
\end{description}
\begin{center}\rule{.85\textwidth}{.01cm}\end{center}
\section*{(pro) TREE\_CLEANUP \label{Tree_Cleanup_lab}}
\addcontentsline{toc}{section}{Tree\_Cleanup}
\textbf{Defined in file:} \newline
\small{C:/cal/Operations/SDI\_Instruments/common/idl/core/edit\_console\_settings.pro} \newline
\textbf{Procedure Documentation:} \\
If this editor was created by the SDI console, alert it that we have closed.\newline
\textbf{Arguments:}
\begin{description}
\vspace{-.15cm}
\item[] \hspace{.5cm} \verb"id": Widget id
\end{description}
\begin{center}\rule{.85\textwidth}{.01cm}\end{center}
\section*{(pro) TREE\_EVENT \label{Tree_Event_lab}}
\addcontentsline{toc}{section}{Tree\_Event}
\textbf{Defined in file:} \newline
\small{C:/cal/Operations/SDI\_Instruments/common/idl/core/edit\_console\_settings.pro} \newline
\textbf{Procedure Documentation:} \\
Handle events generated by the tree widget.\newline
\textbf{Arguments:}
\begin{description}
\vspace{-.15cm}
\item[] \hspace{.5cm} \verb"event": Widget event structure
\end{description}
\begin{center}\rule{.85\textwidth}{.01cm}\end{center}
\section*{(pro) WRITE\_SPECTRA\_NETCDF \label{Write_Spectra_NetCDF_lab}}
\addcontentsline{toc}{section}{Write\_Spectra\_NetCDF}
\textbf{Defined in file:} \newline
\small{C:/cal/Operations/SDI\_Instruments/common/idl/core/write\_spectra\_netcdf.pro} \newline
\textbf{Procedure Documentation:} \\
Wrapper for creating and writing to NETCDF files, used by the Spectrum plugin to save spectral data.\newline
\textbf{Arguments:}
\begin{description}
\vspace{-.15cm}
\item[] \hspace{.5cm} \verb"ncdid": File id to write to, 0 if opening a new file
\vspace{-.15cm}
\item[] \hspace{.5cm} \verb"spectra": The array of spectra (nzones X nchannels
\vspace{-.15cm}
\item[] \hspace{.5cm} \verb"start_time": The time at which the exposure started
\vspace{-.15cm}
\item[] \hspace{.5cm} \verb"end_time": The time at which the exposure finished
\vspace{-.15cm}
\item[] \hspace{.5cm} \verb"nscans": The number of scans in the exposure
\vspace{-.15cm}
\item[] \hspace{.5cm} \verb"acc_im": The accumulated allsky image for the exposure
\vspace{-.15cm}
\item[] \hspace{.5cm} \verb"create=create": Set this to create a new file
\vspace{-.15cm}
\item[] \hspace{.5cm} \verb"fname=fname": Filename of the file
\vspace{-.15cm}
\item[] \hspace{.5cm} \verb"return_id=return_id": When creating a new file, the netcdf id is returned
\vspace{-.15cm}
\item[] \hspace{.5cm} \verb"header=header": Header info, when creating a new file
\vspace{-.15cm}
\item[] \hspace{.5cm} \verb"data=data": Misc data, see function body
\vspace{-.15cm}
\item[] \hspace{.5cm} \verb"reopen=reopen": Reopen a file and append to it, for example after a shutdown
\vspace{-.15cm}
\item[] \hspace{.5cm} \verb"update=update": Open for writing, see function body
\end{description}
\begin{center}\rule{.85\textwidth}{.01cm}\end{center}
\section*{(pro) COMMS\_WRAPPER \label{comms_wrapper_lab}}
\addcontentsline{toc}{section}{comms\_wrapper}
\textbf{Defined in file:} \newline
\small{C:/cal/Operations/SDI\_Instruments/common/idl/core/comms\_wrapper.pro} \newline
\textbf{Procedure Documentation:} \\
No Doc\newline
\textbf{Arguments:}
\begin{description}
\vspace{-.15cm}
\item[] \hspace{.5cm} \verb"port": No Doc
\vspace{-.15cm}
\item[] \hspace{.5cm} \verb"dll_name": No Doc
\vspace{-.15cm}
\item[] \hspace{.5cm} \verb"type=type": No Doc
\vspace{-.15cm}
\item[] \hspace{.5cm} \verb"": No Doc
\end{description}
\begin{center}\rule{.85\textwidth}{.01cm}\end{center}
\section*{(pro) CONSOLE\_CRASH\_ROUTINE \label{console_crash_routine_lab}}
\addcontentsline{toc}{section}{console\_crash\_routine}
\textbf{Defined in file:} \newline
\small{C:/cal/Operations/SDI\_Instruments/common/idl/core/crash\_routines.pro} \newline
\textbf{Procedure Documentation:} \\
Check to see if the console `crash' file is present. If it is, it is likely that the SDI console has stopped running, and this gets logged.\newline
\textbf{Arguments:}
\begin{description}
\vspace{-.15cm}
\item[] \hspace{.5cm} \verb"log_file": The filename to send/append log output to
\end{description}
\begin{center}\rule{.85\textwidth}{.01cm}\end{center}
\section*{(pro) CONSOLE\_MAKE\_CRASH\_FILE \label{console_make_crash_file_lab}}
\addcontentsline{toc}{section}{console\_make\_crash\_file}
\textbf{Defined in file:} \newline
\small{C:/cal/Operations/SDI\_Instruments/common/idl/core/crash\_routines.pro} \newline
\textbf{Procedure Documentation:} \\
Create the console `crash' file.\newline
\textbf{Arguments:}
\begin{description}
\vspace{-.15cm}
\item[] \hspace{.5cm} \verb"crash_file": Filename for the crash file
\end{description}
\begin{center}\rule{.85\textwidth}{.01cm}\end{center}
\section*{(pro) CRASH\_ROUTINES \label{crash_routines_lab}}
\addcontentsline{toc}{section}{crash\_routines}
\textbf{Defined in file:} \newline
\small{C:/cal/Operations/SDI\_Instruments/common/idl/core/crash\_routines.pro} \newline
\textbf{Procedure Documentation:} \\
This gets called by a Windows scheduled script, and checks to see if a crash file is present (the console should delete this file, so if it is present, the console has likely crashed), and if so it logs a crash. If not ,it recreates the file.\newline
Takes no arguments \\
\begin{center}\rule{.85\textwidth}{.01cm}\end{center}
\section*{(pro) DEFINE\_VARIABLES \label{define_variables_lab}}
\addcontentsline{toc}{section}{define\_variables}
\textbf{Defined in file:} \newline
\small{C:/cal/Operations/SDI\_Instruments/common/idl/core/edit\_console\_settings.pro} \newline
\textbf{Procedure Documentation:} \\
Create the SDI variables/structures.\newline
\textbf{Arguments:}
\begin{description}
\vspace{-.15cm}
\item[] \hspace{.5cm} \verb"var_holder": Variables will be returned in this structure
\end{description}
\begin{center}\rule{.85\textwidth}{.01cm}\end{center}
\section*{(pro) DRIVE\_MOTOR\_WAIT\_FOR\_POSITION \label{drive_motor_wait_for_position_lab}}
\addcontentsline{toc}{section}{drive\_motor\_wait\_for\_position}
\textbf{Defined in file:} \newline
\small{C:/cal/Operations/SDI\_Instruments/common/idl/core/drive\_motor.pro} \newline
\textbf{Procedure Documentation:} \\
Wait for a position reached notification from the motor (a `p' character). A timeout can be provided to prevent waiting forever.\newline
\textbf{Arguments:}
\begin{description}
\vspace{-.15cm}
\item[] \hspace{.5cm} \verb"port": Com port for the motor
\vspace{-.15cm}
\item[] \hspace{.5cm} \verb"dll_name": Name of the SDI\_External dll
\vspace{-.15cm}
\item[] \hspace{.5cm} \verb"com": String `com' type, e.g. "moxa"
\vspace{-.15cm}
\item[] \hspace{.5cm} \verb"max_wait_time=max_wait_time": Max time to wait in seconds
\vspace{-.15cm}
\item[] \hspace{.5cm} \verb"errcode=errcode": Returned error code
\end{description}
\begin{center}\rule{.85\textwidth}{.01cm}\end{center}
\section*{(pro) EDIT\_CONSOLE\_SETTINGS \label{edit_console_settings_lab}}
\addcontentsline{toc}{section}{edit\_console\_settings}
\textbf{Defined in file:} \newline
\small{C:/cal/Operations/SDI\_Instruments/common/idl/core/edit\_console\_settings.pro} \newline
\textbf{Procedure Documentation:} \\
Entry point for the console settings editor. Can be called ddirectly from IDL command line, or from the SDI console.\newline
\textbf{Arguments:}
\begin{description}
\vspace{-.15cm}
\item[] \hspace{.5cm} \verb"filename=filename": Pass in a filename to load upon startup
\vspace{-.15cm}
\item[] \hspace{.5cm} \verb"leader=leader": Widget leader, when called from the console
\vspace{-.15cm}
\item[] \hspace{.5cm} \verb"console=console": The console object reference, if started from the console
\end{description}
\begin{center}\rule{.85\textwidth}{.01cm}\end{center}
\section*{(pro) EDIT\_LOAD\_SETTINGS \label{edit_load_settings_lab}}
\addcontentsline{toc}{section}{edit\_load\_settings}
\textbf{Defined in file:} \newline
\small{C:/cal/Operations/SDI\_Instruments/common/idl/core/edit\_console\_settings.pro} \newline
\textbf{Procedure Documentation:} \\
Load a settings file from disk.\newline
\textbf{Arguments:}
\begin{description}
\vspace{-.15cm}
\item[] \hspace{.5cm} \verb"filename=filename": Filename to load
\end{description}
\begin{center}\rule{.85\textwidth}{.01cm}\end{center}
\section*{(pro) EDIT\_PORT\_SETTINGS \label{edit_port_settings_lab}}
\addcontentsline{toc}{section}{edit\_port\_settings}
\textbf{Defined in file:} \newline
\small{C:/cal/Operations/SDI\_Instruments/common/idl/core/edit\_console\_settings.pro} \newline
\textbf{Procedure Documentation:} \\
Create an xvaredit dialog for editing the port structure.\newline
Takes no arguments \\
\begin{center}\rule{.85\textwidth}{.01cm}\end{center}
\section*{(pro) EDIT\_SAVE\_SETTINGS \label{edit_save_settings_lab}}
\addcontentsline{toc}{section}{edit\_save\_settings}
\textbf{Defined in file:} \newline
\small{C:/cal/Operations/SDI\_Instruments/common/idl/core/edit\_console\_settings.pro} \newline
\textbf{Procedure Documentation:} \\
Save the current settings.\newline
\textbf{Arguments:}
\begin{description}
\vspace{-.15cm}
\item[] \hspace{.5cm} \verb"filename=filename": Filename to save to
\vspace{-.15cm}
\item[] \hspace{.5cm} \verb"nosplash=nosplash": Optionally hide the "File saved" dialog
\end{description}
\begin{center}\rule{.85\textwidth}{.01cm}\end{center}
\section*{(pro) GET\_JD0\_SEC \label{get_jd0_sec_lab}}
\addcontentsline{toc}{section}{get\_jd0\_sec}
\textbf{Defined in file:} \newline
\small{C:/cal/Operations/SDI\_Instruments/common/idl/core/get\_jd0\_sec.pro} \newline
\textbf{Procedure Documentation:} \\
Get the current julian date and the seconds into the day.\newline
\textbf{Arguments:}
\begin{description}
\vspace{-.15cm}
\item[] \hspace{.5cm} \verb"jd0": OUT: Julian date at midnight I think...
\vspace{-.15cm}
\item[] \hspace{.5cm} \verb"sec": OUT: Seconds into the julian day
\end{description}
\begin{center}\rule{.85\textwidth}{.01cm}\end{center}
\section*{(pro) LOAD\_PAL \label{load_pal_lab}}
\addcontentsline{toc}{section}{load\_pal}
\textbf{Defined in file:} \newline
\small{C:/cal/Operations/SDI\_Instruments/common/idl/core/load\_pal.pro} \newline
\textbf{Procedure Documentation:} \\
No Doc\newline
\textbf{Arguments:}
\begin{description}
\vspace{-.15cm}
\item[] \hspace{.5cm} \verb"culz": No Doc
\vspace{-.15cm}
\item[] \hspace{.5cm} \verb"idl_table=itbl": No Doc
\vspace{-.15cm}
\item[] \hspace{.5cm} \verb"bright=brt": No Doc
\vspace{-.15cm}
\item[] \hspace{.5cm} \verb"proportion=prp": No Doc
\end{description}
\begin{center}\rule{.85\textwidth}{.01cm}\end{center}
\section*{(pro) PAL\_SUBSAMP \label{pal_subsamp_lab}}
\addcontentsline{toc}{section}{pal\_subsamp}
\textbf{Defined in file:} \newline
\small{C:/cal/Operations/SDI\_Instruments/common/idl/core/load\_pal.pro} \newline
\textbf{Procedure Documentation:} \\
No Doc\newline
\textbf{Arguments:}
\begin{description}
\vspace{-.15cm}
\item[] \hspace{.5cm} \verb"idxlo": No Doc
\vspace{-.15cm}
\item[] \hspace{.5cm} \verb"idxhi": No Doc
\vspace{-.15cm}
\item[] \hspace{.5cm} \verb"sred": No Doc
\vspace{-.15cm}
\item[] \hspace{.5cm} \verb"sgrn": No Doc
\vspace{-.15cm}
\item[] \hspace{.5cm} \verb"sblu": No Doc
\vspace{-.15cm}
\item[] \hspace{.5cm} \verb"brt": No Doc
\vspace{-.15cm}
\item[] \hspace{.5cm} \verb"satval": No Doc
\vspace{-.15cm}
\item[] \hspace{.5cm} \verb"sign": No Doc
\end{description}
\begin{center}\rule{.85\textwidth}{.01cm}\end{center}
\section*{(pro) RESTART\_MOXA \label{restart_moxa_lab}}
\addcontentsline{toc}{section}{restart\_moxa}
\textbf{Defined in file:} \newline
\small{C:/cal/Operations/SDI\_Instruments/common/idl/core/restart\_moxa.pro} \newline
\textbf{Procedure Documentation:} \\
Restart the MOXA USB hub, using pstools (TODO: is this used? Paths are hard coded...)\newline
Takes no arguments \\
\begin{center}\rule{.85\textwidth}{.01cm}\end{center}
\section*{(pro) SCHEDULE\_READER \label{schedule_reader_lab}}
\addcontentsline{toc}{section}{schedule\_reader}
\textbf{Defined in file:} \newline
\small{C:/cal/Operations/SDI\_Instruments/common/idl/core/schedule\_reader.pro} \newline
\textbf{Procedure Documentation:} \\
Query an SDI schedule file for the next command.\newline
\textbf{Arguments:}
\begin{description}
\vspace{-.15cm}
\item[] \hspace{.5cm} \verb"schedule_file": Schedule file name
\vspace{-.15cm}
\item[] \hspace{.5cm} \verb"schedule_line": The current schedule line
\vspace{-.15cm}
\item[] \hspace{.5cm} \verb"xcomm": OUT: string command
\vspace{-.15cm}
\item[] \hspace{.5cm} \verb"xargs": OUT: string array of arguments
\vspace{-.15cm}
\item[] \hspace{.5cm} \verb"lat": Geographic latitude
\vspace{-.15cm}
\item[] \hspace{.5cm} \verb"lon": Geographic longitude
\vspace{-.15cm}
\item[] \hspace{.5cm} \verb"console_ref": Object reference for the console
\vspace{-.15cm}
\item[] \hspace{.5cm} \verb"refresh_nm_per_step=refresh_nm_per_step": Look for a nm per step refresh command (special syntax)
\vspace{-.15cm}
\item[] \hspace{.5cm} \verb"refresh_phasemap=refresh_phasemap": Look for a phasemap refresh command (special syntax)
\end{description}
\begin{center}\rule{.85\textwidth}{.01cm}\end{center}
\end{document}
